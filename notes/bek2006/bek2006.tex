%-----------------------------------------------------------------------------%
%Packages%
\documentclass[10pt, a4paper, twoside]{article}
\usepackage{amsmath, amsfonts, listings, amssymb, mathtools, amsthm} %Mathematical Expressions package
\usepackage{mathtools}
\usepackage[usenames, dvipsnames]{color} %Color naming packages
\usepackage[margin=1.5cm]{geometry}
\usepackage{float}
\usepackage{verbatim} %for code
\usepackage[pdftex]{graphics}
\usepackage{ulem}
\usepackage{hyperref}
\usepackage{tikz}

\usetikzlibrary{arrows,shapes}

%Graphis Extensions
\DeclareGraphicsExtensions{.png, .jpg}
\parindent 0pt

% Predefined things such as commands, etc.

\newcommand{\aRel}[1] {
  \sim_{\mathcal{#1} }
}

\newcommand{\kripkeFrame}[2] {
  (#1, \aRel{#2})
}

\newcommand{\kripkeModel}[3] {
  (#1, \aRel{#2}, #3)
}

\newcommand{\frKripModel}[2] { % defined via Kripke Frame + valuation
  (#1, #2)
}

\newcommand{\actModel}[3]{
  (#1, \aRel{#2}, #3)
}

\newcommand{\frActModel}[2] { % defined via Kripke Frame + Pre
  (#1, #2)
}

\newcommand{\note}[1]{\textsc{\textbf{#1}}}
\newcommand{\Universal}{$\mathcal{U}$}
\newcommand{\modalLog}{$\mathcal{L}$}
\newcommand{\modLogInf}{$\mathcal{L}_\inf$}
\newcommand{\epActLog}{\modalLog$([\alpha])$}
\newcommand{\epActLogCommonKnowledge}{\modalLog$([\alpha],\box^{*})$}

\newtheorem{defn}{Definition}
\newtheorem{thm}{Theorem}
\newtheorem{lemma}{Lemma}
\newtheorem*{remrk}{Remark}

% Drawings of frames

\tikzstyle{vertex}=[circle,fill=black!25,minimum size=20pt,inner sep=0pt]
\tikzstyle{selected vertex} = [vertex, fill=red!24]
\tikzstyle{edge} = [draw,thick,-]
\tikzstyle{weight} = [font=\small]

%-----------------------------------------------------------------------------%
%Document%
\begin{document}
Citations: \\
\\
van Benthem, van Eijek and Kooi (BEK).
They state that they want to deal with common knowledge being an opaque idea for
multi-agent communication.
They propose new systems extending the epistemic base language with
``relativized common knowledge".
They show that their systems can also deal with factual change and alteration
rather than just information change.\\
\\
Their main technical result is what appears to be PDL interpreted epistemically.
The claim
\begin{quote}
	This is a dynamic epistemic logic $\ldots$ capable of expressing all
	model-shifting operations with finite action models while providing a
	compositional analysis for a wide range of informational events.
\end{quote}
Their results appear to be mainly stuck in $S5$ and $KD45$.
So there really isn't much generality to $K$ at the moment.\\
\\
Some limitations they mention:
\begin{itemize}
	\item an update model with several events as a disjunction of instructions
	under conditions
	\item suggests update models are built from simple actions by means of regular
	operations of choice, sequence and iteration
	\item perhaps even concurrent composition of events
	\item See Miller and Moss (23) of this paper!
	\item atomic form of post conditions for world changing actions
\end{itemize}
Relativized Common Knowledge seems suspiciously similar to what BMS1998
discusses.\\
\\
See Section 4.2 for PDL interpreted epistemically.
See Section 4.3!
Is this a PDL-like execution of a pointed model?
This seems exactly like BMS1998 though...?\\
\\
Section 5 is like a massive motivational thing.
It's nice that they have a PDL style framework though...\\
\\
Section 6 is interesting.
In particular BEK ask about Program constructions over update models.
They obtain a calculus describing effects of more complex events.
There is a paper that makes PAL undecidable, but partial axiomatizations might
be possible.
\end{document}
