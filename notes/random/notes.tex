%-----------------------------------------------------------------------------%
%Packages%
\documentclass[12pt, a4paper, titlepage]{scrartcl}
\usepackage{amsmath, amsfonts, listings, amssymb, mathtools, amsthm} %Mathematical Expressions package
\usepackage{mathtools}
\usepackage[usenames, dvipsnames]{color} %Color naming packages
\usepackage[margin=1.5cm]{geometry}
\usepackage{float}
\usepackage{verbatim} %for code
\usepackage[pdftex]{graphics}
\usepackage{hyperref}
\usepackage{cleveref}
\usepackage{tikz}
\usepackage{comment}
\usepackage[nottoc]{tocbibind}
%\usepackage[square]{natbib}
\usepackage{caption}
\usepackage{subcaption}

\addtokomafont{disposition}{\rmfamily}
\usetikzlibrary{arrows,shapes}

%Graphis Extensions
\DeclareGraphicsExtensions{.png, .jpg}
\parindent 0pt

% Predefined things such as commands, etc.

\newcommand{\aRel}[1] {
  \sim_{#1} 
}

\newcommand{\kripkeFrame}[2] {
  (#1, \aRel{#2})
}

\newcommand{\kripkeModel}[3] {
  (#1, \aRel{#2}, #3)
}

\newcommand{\frKripModel}[2] { % defined via Kripke Frame + valuation
  (#1, #2)
}

\newcommand{\actModel}[3]{
  (#1, \aRel{#2}, #3)
}

\newcommand{\actModelStates}[4] {
	(#1, \aRel{#2}, #3, #4)
}

\newcommand{\frActModel}[2] { % defined via Kripke Frame + Pre
  (#1, #2)
}

\newcommand{\note}[1]{\textsc{\textbf{#1}}}
\newcommand{\Universal}{$\mathcal{U}$}
\newcommand{\modalLog}{$\mathcal{L}$}
\newcommand{\modLogInf}{$\mathcal{L}_\inf$}
\newcommand{\epActLog}{\modalLog$([\alpha])$}
\newcommand{\epActLogCommonKnowledge}{\modalLog$([\alpha],\box^{*})$}

\newtheorem{defn}{Definition}
\newtheorem{thm}{Theorem}
\newtheorem{lemma}{Lemma}
\newtheorem{corr}{Corrollary}
\newtheorem*{remrk}{Remark}

% Drawings of frames

\tikzstyle{vertex}=[circle,fill=black!25,minimum size=20pt,inner sep=0pt]
\tikzstyle{selected vertex} = [vertex, fill=red!24]
\tikzstyle{edge} = [draw,thick,->]
\tikzstyle{weight} = [font=\small]

%-----------------------------------------------------------------------------%
%Document%
\begin{document}

\section{Definitions}

\begin{defn} \label{frame}
	Let $\Sigma$ be a set and $\aRel{A}$ a set indexed by $A$, a set of agents.
	Then the tuple $\kripkeFrame{\Sigma}{A}$ is a frame.
\end{defn}

A frame represents a mathematical structure, but holds no meaningful information.
In order to give meaning and be able to draw conclusions from the frame, we can
construct a valuation function that transforms the frame into a model detailing
information.
\begin{defn} \label{model}
	Let $F = \kripkeFrame{\Sigma}{A}$ be a frame, and $A$ a set of agents.
	Let $V: \mathbb{P} \to \mathcal{P}(\Sigma)$ be a function mapping any
	well-formed	sentence in $\mathbb{P}$ to a subset of $\Sigma$.
	We say $V$ is a valuation function on $F$.\\
	\\
	Then $M = \frKripModel{F}{V} = \kripkeModel{\Sigma}{A}{V}$ is a Kripke
	model.
\end{defn}

Our models are now informative models that we can reason with.
In an epistemic logic context, we say that each $s \in \Sigma$ is a possible
world.
Each relation in $\aRel{A}$ is an accessibility relation, such that $s_1 \sim_a
s_2$ signifies an agent $a \in A$ considering that, if $s_1$ was the ``current"
world $a$ would believe $s_2$ was possible.\\
\\
We define an event model as a frame with a precondition function, and extend it
to take a set of 
{\em More stuff follows: some things about event models, etc.}\\
\\
Let us consider the following operation

\begin{defn} \label{gen_union_1}
	Let $M_1 = \actModelStates{\Sigma_1}{A}{pre_1}{\Gamma_1}$ and $M_2 =
	\actModelStates{\Sigma_2}{A}{pre_2}{\Gamma_2}$ be two event models with
	designated states.\\
	\\
	Let $M_1 \sqcup_B M_2 = \actModelStates{\Sigma}{A}{pre}{\Gamma}$ be the disjoint
	union of $M_1$ and $M_2$ for some group	of agents $B \subseteq A$, such that
	\begin{itemize}
		\item $\Sigma = \Sigma_1 \sqcup \Sigma_2$
		\item $\aRel{A} = {\aRel{A}}_1 \sqcup {\aRel{A}}_2 \sqcup \{w \sim_b w' | b \in
		B \land (w \in \Gamma_1 \land w' \in \Gamma_2 \lor w \in \Gamma_2 \land w' \in \Gamma_1)\}$
		\item $pre(w) = pre_1(w)$ if $w \in \Sigma_1$ and $pre(w) = pre_2(w)$ if $w
		\in \Sigma_2$
		\item $\Gamma = \Gamma_1 \sqcup \Gamma_2$
	\end{itemize}
\end{defn}

\begin{lemma}
	The algebra (?)
	\[
		\phi = \phi ; \theta | \phi ^{\ast} | \phi \sqcup \theta | \text{announce} |
		\text{skip} | \text{crash}
	\]
	generates all event models in {\bf K45}.
\end{lemma}

\begin{note}
	Actually this isn't true.
	As an example, you can generate a model outside of {\bf K45} and into ???.\\
\end{note}
\\
Let $L$ be the language of {\bf K45}.
We say $\mathcal{AM}$ is the class of all action models over $L$.
Then we say $\mathcal{AM}_{PA}$ is the class of all action models that
are public announcements of a sentence in $L$.\\
\\
Let us define the following action models
\begin{defn} \label{insanity}
Let $M = \actModel{\{\sigma\}}{\varnothing}{{\sigma, \phi}}$ be an action model
where $\phi \in L$.
Then $\mathcal{AM}_{IN} = \{M | \phi \in L\}$ is the class of all action models
that represent the onset of insanity.
\end{defn}

We define an operation
\begin{defn} \label{believe}
Let $M_1, M_2 \in \mathcal{AM}$ and without loss of generality let $M_1$, $M_2$
be disjoint.
Let $A$ be the agents for an action model $M_1$ and $M_2$.
Let $B \subseteq A$.
Then $M_1 \rightarrow_B M_2 = \actModel{\Sigma}{}{pre}$ where
\begin{itemize}
  \item $\Sigma = \Sigma_1 \cup \Sigma_2$
  \item $\sim = \sim_1 \cup \sim_2 \cup \{(a, b) | a \in \Sigma_1 \land b \in
  \Sigma_2 \}$
  \item $pre = pre_1 \cup pre_2$
\end{itemize}
\end{defn}
$M = M_1 \rightarrow_B M_2$ is the operation such that after $M$'s execution, $B$ will
believe $M_2$ is executed if any action in $M_1$ is executed.\\
\\
\begin{lemma} \label{gen_trees}
Suppose $\phi$ is a formula and from $L_{\otimes \forall} \exists_B \phi$.
Then $M \in \mathcal{AM}$ and after $M$'s execution $\models \phi$, the language
\[
  \phi = \phi ; \theta | \phi ^ {\ast} | \phi \rightarrow_b \theta | skip |
  crash | announce | insanity
\]
will generate a model $M'$ that is bisimilar to $M$.
\end{lemma}

We adopt the convention for pointed event models that $M^1 = \{\Sigma^1, \sim^1,
pre^1, T^1\}$ and $M^\alpha = \{\Sigma^\alpha,\sim^\alpha,pre^\alpha,T^\alpha\}$.

Let us redefine our operations in terms of pointed models:
\begin{defn} \label{possOne}
Let $M_1$ and $M_2$ be multi-pointed event models, and without loss of
generality let $M_1$ and $M_2$ be disjoint. (?)
Let $B$ be a subset of agents participant in $M_1$ or $M_2$.
We define $M = M_1 \to_B M_2$ as 
\begin{itemize}
  \item $\Sigma = \Sigma^1 \cup \Sigma^2$
  % TODO: change this to a matrix union environment
  \item $\forall k \in A \sim_k = \sim^1_k \cup \sim^2_k \cup \{(a, b) | a \in \Sigma^1 \land b \in
  \Sigma^2 \land a \in T^1 \land b \in T^2 \land k \in B\}$
  \item $pre = pre^1 \cup pre^2$
  \item $T = T^1$
\end{itemize}
\end{defn}

This indicates that from $M_1$, an agent in $B$ considers $M_2$ a possible
update.\\
\\

Another operation:
\begin{defn} \label{possTwo}
Let $M_1$ and $M_2$ be multi-pointed event models, and without loss of
generality let $M_1$ and $M_2$ be disjoint. (?)
Let $B$ be a subset of agents participant in $M_1$ or $M_2$.
We define $M = M_1 \mapsto_B M_2$ as 
\begin{itemize}
  \item $\Sigma = \Sigma^1 \cup \Sigma^2$
  % TODO: change this to a matrix union environment
  \item $\forall k \in B \implies \sim_k = \sim^1_k \setminus \{ (a,b) \} \cup \sim^2_k \cup \{(a, b) | a \in \Sigma^1 \land b \in
  \Sigma^2 \land a \in T^1 \land b \in T^2 \land k \in B \}$
  \item $pre = pre^1 \cup pre^2$
  \item $T = T^1$
\end{itemize}
\end{defn}

The ``meaning" of $M_1 \mapsto_B M_2$ would be that given a set of actions in
$M_1$ the agents in $B$ believe that one of a set of actions in $M_2$ was
executed instead.\\
\\
It would be desirable to see whether the class of public announcements, with
$\to$ and $\mapsto$ would be sufficient to construct all tree-like action
models.
Conversely, using insanity and public announcement event models with $\to$ should be sufficient to construct
all tree-like action models.

\begin{thm}
Let $M$ be a tree-like event model.
$M$ can be constructed by public announcements, $\to$ and $\mapsto$.
\end{thm}
To prove this we shall show that any pointed subtree of $M$, $M'$ can be
constructed by public announcements, $\to$ and $\mapsto$.
We will induct on the subtrees of $M$.\\
\\
Consider $M'$, a subtree of $M$.
Suppose $M'$ is a leaf node, and thus it must be a public announcement, with ints
single point.
Then our induction hypothesis holds for leaf nodes of $M$.\\
\\
Suppose $M'$ is an arbitrary subtree of $M$.
Let $K^1$, $K^2$, $\ldots$, $K^n$ be the subtrees of $M$ whose parent node is
the root node of $M'$.
Furthermore, let us say that each of $K^1, K^2, \ldots, K^n$ fulfills the
induction hypothesis.
Consider the public announcement $P = \{\Sigma,\sim,pre,T\}$, such that
\begin{itemize}
  \item $\Sigma = \{ \sigma \}$
  \item $\forall a \in A \implies \sim_a = \{ (\sigma,\sigma) \}$
  \item $pre = \{ \(\sigma, \phi\}$
  \item $T = \{\sigma\}$
\end{itemize}
where $\phi$ is the precondition of the root node of $M'$.
Then $M = (((P \mapsto K^1) \to K^2) \to \ldots \to K^n$.
\end{document}
