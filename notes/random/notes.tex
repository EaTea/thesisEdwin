%-----------------------------------------------------------------------------%
%Packages%
\documentclass[12pt, a4paper, titlepage]{scrartcl}
\usepackage{amsmath, amsfonts, listings, amssymb, mathtools, amsthm} %Mathematical Expressions package
\usepackage{mathtools}
\usepackage[usenames, dvipsnames]{color} %Color naming packages
\usepackage[margin=1.5cm]{geometry}
\usepackage{float}
\usepackage{verbatim} %for code
\usepackage[pdftex]{graphics}
\usepackage{hyperref}
\usepackage{cleveref}
\usepackage{tikz}
\usepackage{comment}
\usepackage[nottoc]{tocbibind}
%\usepackage[square]{natbib}
\usepackage{caption}
\usepackage{subcaption}

\addtokomafont{disposition}{\rmfamily}
\usetikzlibrary{arrows,shapes}

%Graphis Extensions
\DeclareGraphicsExtensions{.png, .jpg}
\parindent 0pt

% Predefined things such as commands, etc.

\newcommand{\aRel}[1] {
  \sim_{#1} 
}

\newcommand{\kripkeFrame}[2] {
  (#1, \aRel{#2})
}

\newcommand{\kripkeModel}[3] {
  (#1, \aRel{#2}, #3)
}

\newcommand{\frKripModel}[2] { % defined via Kripke Frame + valuation
  (#1, #2)
}

\newcommand{\actModel}[3]{
  (#1, \aRel{#2}, #3)
}

\newcommand{\actModelStates}[4] {
	(#1, \aRel{#2}, #3, #4)
}

\newcommand{\frActModel}[2] { % defined via Kripke Frame + Pre
  (#1, #2)
}

\newcommand{\note}[1]{\textsc{\textbf{#1}}}
\newcommand{\Universal}{$\mathcal{U}$}
\newcommand{\modalLog}{$\mathcal{L}$}
\newcommand{\modLogInf}{$\mathcal{L}_\inf$}
\newcommand{\epActLog}{\modalLog$([\alpha])$}
\newcommand{\epActLogCommonKnowledge}{\modalLog$([\alpha],\box^{*})$}

\newtheorem{defn}{Definition}
\newtheorem{thm}{Theorem}
\newtheorem{lemma}{Lemma}
\newtheorem{corr}{Corrollary}
\newtheorem*{remrk}{Remark}

% Drawings of frames

\tikzstyle{vertex}=[circle,fill=black!25,minimum size=20pt,inner sep=0pt]
\tikzstyle{selected vertex} = [vertex, fill=red!24]
\tikzstyle{edge} = [draw,thick,->]
\tikzstyle{weight} = [font=\small]

%-----------------------------------------------------------------------------%
%Document%
\begin{document}

\section{Definitions}

\begin{defn} \label{frame}
	Let $\Sigma$ be a set and $\aRel{A}$ a set indexed by $A$, a set of agents.
	Then the tuple $\kripkeFrame{\Sigma}{A}$ is a frame.
\end{defn}

A frame represents a mathematical structure, but holds no meaningful information.
In order to give meaning and be able to draw conclusions from the frame, we can
construct a valuation function that transforms the frame into a model detailing
information.
\begin{defn} \label{model}
	Let $F = \kripkeFrame{\Sigma}{A}$ be a frame, and $A$ a set of agents.
	Let $V: \mathbb{P} \to \mathcal{P}(\Sigma)$ be a function mapping any
	well-formed	sentence in $\mathbb{P}$ to a subset of $\Sigma$.
	We say $V$ is a valuation function on $F$.\\
	\\
	Then $M = \frKripModel{F}{V} = \kripkeModel{\Sigma}{A}{V}$ is a Kripke
	model.
\end{defn}

Our models are now informative models that we can reason with.
In an epistemic logic context, we say that each $s \in \Sigma$ is a possible
world.
Each relation in $\aRel{A}$ is an accessibility relation, such that $s_1 \sim_a
s_2$ signifies an agent $a \in A$ considering that, if $s_1$ was the ``current"
world $a$ would believe $s_2$ was possible.\\
\\
We define an event model as a frame with a precondition function, and extend it
to take a set of 
{\em More stuff follows: some things about event models, etc.}\\
\\
Let us consider the following operation

\begin{defn} \label{gen_union_1}
	Let $M^1 = \actModelStates{\Sigma_1}{A}{pre_1}{\Gamma_1}$ and $M^2 =
	\actModelStates{\Sigma_2}{A}{pre_2}{\Gamma_2}$ be two event models with
	designated states.\\
	\\
	Let $M^1 \sqcup_B M^2 = \actModelStates{\Sigma}{A}{pre}{\Gamma}$ be the disjoint
	union of $M^1$ and $M^2$ for some group	of agents $B \subseteq A$, such that
	\begin{itemize}
		\item $\Sigma = \Sigma_1 \sqcup \Sigma_2$
		\item $\aRel{A} = {\aRel{A}}_1 \sqcup {\aRel{A}}_2 \sqcup \{w \sim_b w' | b \in
		B \land (w \in \Gamma_1 \land w' \in \Gamma_2 \lor w \in \Gamma_2 \land w' \in \Gamma_1)\}$
		\item $pre(w) = pre_1(w)$ if $w \in \Sigma_1$ and $pre(w) = pre_2(w)$ if $w
		\in \Sigma_2$
		\item $\Gamma = \Gamma_1 \sqcup \Gamma_2$
	\end{itemize}
\end{defn}

\begin{lemma}
	The algebra (?)
	\[
		\phi = \phi ; \theta | \phi ^{\ast} | \phi \sqcup \theta | \text{announce} |
		\text{skip} | \text{crash}
	\]
	generates all event models in {\bf K45}.
\end{lemma}

\begin{note}
	Actually this isn't true.
	As an example, you can generate a model outside of {\bf K45} and into ???.\\
\end{note}
\\
Let $L$ be the language of {\bf K45}.
We say $\mathcal{AM}$ is the class of all action models over $L$.
Then we say $\mathcal{AM}_{PA}$ is the class of all action models that
are public announcements of a sentence in $L$.\\
\\
Let us define the following action models
\begin{defn} \label{insanity}
Let $M = \actModel{\{\sigma\}}{\varnothing}{{\sigma, \phi}}$ be an action model
where $\phi \in L$.
Then $\mathcal{AM}_{IN} = \{M | \phi \in L\}$ is the class of all action models
that represent the onset of insanity.
\end{defn}

We define an operation
\begin{defn} \label{believe}
Let $M^1, M^2 \in \mathcal{AM}$ and without loss of generality let $M^1$, $M^2$
be disjoint.
Let $A$ be the agents for an action model $M^1$ and $M^2$.
Let $B \subseteq A$.
Then $M^1 \rightarrow_B M^2 = \actModel{\Sigma}{}{pre}$ where
\begin{itemize}
  \item $\Sigma = \Sigma_1 \cup \Sigma_2$
  \item $\sim = \sim_1 \cup \sim_2 \cup \{(a, b) | a \in \Sigma_1 \land b \in
  \Sigma_2 \}$
  \item $pre = pre_1 \cup pre_2$
\end{itemize}
\end{defn}
$M = M^1 \rightarrow_B M^2$ is the operation such that after $M$'s execution, $B$ will
believe $M^2$ is executed if any action in $M^1$ is executed.\\
\\
\begin{lemma} \label{gen_trees}
Suppose $\phi$ is a formula and from $L_{\otimes \forall} \exists_B \phi$.
Then $M \in \mathcal{AM}$ and after $M$'s execution $\models \phi$, the language
\[
  \phi = \phi ; \theta | \phi ^ {\ast} | \phi \rightarrow_b \theta | skip |
  crash | announce | insanity
\]
will generate a model $M'$ that is bisimilar to $M$.
\end{lemma}

We adopt the convention for pointed event models that $M^1 = \{\Sigma^1, \sim^1,
pre^1, T^1\}$ and $M^\alpha = \{\Sigma^\alpha,\sim^\alpha,pre^\alpha,T^\alpha\}$.

%TODO: Add in citations
In FIXME, Hales demonstrates that it is possible to determine whether an
finite epistemic goal can be achieved.
Furthermore, if it is achievable Hales constructs event models that will realise
that goal.
We will focus on defining a framework that constructs these event models in a
recursive manner.
This facilitates an automated method for updating knowledge models to reflect a
knowledge state.\\
\\
The event models that Hales employs resemble finite trees.
We give a rigourous definition for them as follows
\begin{defn} \label{finTree}
%TODO: Add in the definition of a finite tree-like event model.
\end{defn}

\begin{defn} \label{possOne}
Let $M^1$ and $M^2$ be multi-pointed event models, and without loss of
generality let $M^1$ and $M^2$ be disjoint.
Let $B$ be a subset of agents participant in $M^1$ or $M^2$.
We define $M = M^1 \to_B M^2$ as 
\begin{itemize}
  \item $\Sigma = \Sigma^1 \cup \Sigma^2$
  \item $\sim_a =
  \begin{cases}
    \sim^1_a \cup \sim^2_a & \text{if } a \notin C \\
    \displaystyle\bigcup \begin{array}{c}
      \sim^1_a \\
      \sim^2_a \\
      \{(s,t) | s \in T^1 \land t \in T^2 \}
    \end{array} & \text{if } a \in C 
  \end{cases}$
  \item $pre = pre^1 \cup pre^2$
  \item $T = T^1$
\end{itemize}
\end{defn}

$M^1 \to_B M^2$ indicates that from $M^1$, an agent in $B$ considers $M^2$ a possible
update.

%TODO: Wait what is a group announcement? Isn't the null-group the equivalent of
%an insanity model?

\begin{thm}
Let $M$ be a finite tree-like event model.
$M$ can be constructed by public announcements, insanity event models and $\to$.
\end{thm}
To prove this we shall show that any pointed subtree of $M$, $M'$, whose point
is at its root can be constructed by group announcements, insanity event models
and $\to$.
We will induct on the subtrees of $M$.\\
\\
Suppose $M'$ is a leaf node of $M$, $m$.
The leaf nodes of a tree-like event model resemble are public announcements.
Let $M'$ be a pointed public announcement event model with its precondition the
same as $m$.
Then our induction hypothesis holds for $M'$ being a leaf node of $M$.\\
\\
Now, suppose $M'$ is an arbitrary subtree of $M$.
Let $K^1$, $K^2$, $\ldots$, $K^n$ be the subtrees of $M'$ whose parent node is
the root node of $M'$.
Let the group of agents $A^i$ have relations from the root node of $M'$ to a
subtree $K^i$.
Furthermore, let us say that each of $K^1, K^2, \ldots, K^n$ fulfills the
induction hypothesis.
Consider the insanity model $P = (\Sigma,\varnothing,pre,T)$, such that
\begin{itemize}
  \item $\Sigma = \{ \sigma \}$
  \item $pre = \{ (\sigma, \phi)\}$
  \item $T = \{\sigma\}$
\end{itemize}
where $\Sigma$, the set of states of $P$ is the singleton set $\sigma$ and 
$\phi$ is the precondition of the root node of $M'$.
Then $M' = ((\ldots((P \to_{A^1} K^1) \to_{A^2} K^2) \to_{A^3} \ldots)\to_{A^n} K^n)$.
Thus the induction hypothesis holds for $M'$.\\
\\
``Insanity models" are a somewhat cumbersome construct, since their only purpose
is to serve as intermediate nodes in the tree.
They have no meaning outside of a tree-like structure and their semantics are
unintuitive within the context of knowledge or belief.
It would thus be desirable to reduce our atomic models to only group
announcements, since group announcements have more meaning within the context of
knowledge.\\
\\
In order to do so, we must define an operation that ensures that if $M^1$ was to
be executed, a group of agents in $B$ would believe $M^2$ would be executed
instead.
%TODO: Add in the story/motivation --- why is this useful
\begin{defn} \label{possTwo}
Let $M^1$ and $M^2$ be multi-pointed event models, and without loss of
generality let $M^1$ and $M^2$ be disjoint.
Let $B$ be a subset of agents participant in $M^1$ or $M^2$.
We define $M = M^1 \mapsto_B M^2$ as 
\begin{itemize}
  \item $\Sigma = \Sigma^1 \cup \Sigma^2$
  \item $\sim_a =
  \begin{cases}
    \sim^1_a \cup \sim^2_a & \text{if } a \notin B \\
    %FIXME: Make this neater
    \bigcup \begin{array}{c}
      \sim^1_a \setminus \{(s,t) | s \in T^1 \} \\
      \sim^2_a \\
      \{(s,t) | s \in T^1 \land t \in T^2 \}
    \end{array} & \text{if } a \in B 
  \end{cases}$
  \item $pre = pre^1 \cup pre^2$
  \item $T = T^1$
\end{itemize}
\end{defn}

The ``meaning" of $M^1 \mapsto_B M^2$ would be that given a set of actions in
$M^1$ the agents in $B$ believe that one of a set of actions in $M^2$ was
executed instead.\\
\\
It would be desirable to see whether the class of public announcements, with
$\to$ and $\mapsto$ would be sufficient to construct all tree-like action
models.
Conversely, using insanity and public announcement event models with $\to$ should be sufficient to construct
all tree-like action models.
\begin{thm}
Let $M$ be a finite tree-like event model.
$M$ can be constructed by public announcements, $\to$ and $\mapsto$.
\end{thm}
To prove this we shall show that any pointed subtree of $M$, $M'$, whose point
is at its root can be constructed by public announcements, $\to$ and $\mapsto$.
We will induct on the subtrees of $M$.\\
\\
Suppose $M'$ is a leaf node of $M$, $m$.
The leaf nodes of a tree-like event model resemble are public announcements.
Let $M'$ be a pointed public announcement event model with its precondition the
same as $m$.
Then our induction hypothesis holds for $M'$ being a leaf node of $M$.\\
\\
Now, suppose $M'$ is an arbitrary subtree of $M$.
Let $K^1$, $K^2$, $\ldots$, $K^n$ be the subtrees of $M'$ whose parent node is
the root node of $M'$.
%TODO: tidy up the agent relations...?
Let the group of agents $A^i$ have relations from the root node of $M'$ to a
subtree $K^i$.
Furthermore, let us say that each of $K^1, K^2, \ldots, K^n$ fulfills the
induction hypothesis.
Consider the public announcement $P = \{\Sigma,\sim_{A^1},pre,T\}$, such that
\begin{itemize}
  \item $\Sigma = \{ \sigma \}$
  \item $\forall a \in A^1 \implies \sim_a = \{ (\sigma,\sigma) \}$
  \item $pre = \{ (\sigma, \phi)\}$
  \item $T = \{\sigma\}$
\end{itemize}
where $\Sigma$, the set of states of $P$ is the singleton set $\sigma$ and 
$\phi$ is the precondition of the root node of $M'$.
Then $M' = ((\ldots((P \mapsto_{A^1} K^1) \to_{A^2} K^2) \to_{A^3}
      \ldots)\to_{A^n} K^n)$.\\
\\
%TODO: break things up more
It is clear that adding disjoint union to either of these sets of operators
allows us to form all forests of tree-like event models.
It also provides us with an alternate way to form tree-like event models.\\
\\
%TODO: 
However, we cannot form tree-like event models with just $\mapsto$, or even
$\mapsto$ and disjoint union.
Consider $M = ( \{\sigma_1, \sigma_2, \sigma_3, \sigma_4\}, \{\sim_a =
\{(\sigma_1,\sigma_2),(\sigma_1,\sigma_3),(\sigma_2,\sigma_2),(\sigma_3,\sigma_3)\},
\sim_b =
\{(\sigma_1,\sigma_3),(\sigma_1,\sigma_4),(\sigma_3,\sigma_3),(\sigma_4,\sigma_4)\}\},
pre)$.
We cannot form this model using $ \mapsto $, due to $\sim_a \cap \sim_b \neq
\varnothing$.\\
\\
We can form other interesting models besides trees using these operators.
\end{document}
