%-----------------------------------------------------------------------------%
%Packages%
\documentclass[12pt, a4paper, titlepage]{scrartcl}
\usepackage{amsmath, amsfonts, listings, amssymb, mathtools, amsthm} %Mathematical Expressions package
\usepackage{mathtools}
\usepackage[usenames, dvipsnames]{color} %Color naming packages
\usepackage[margin=1.5cm]{geometry}
\usepackage{float}
\usepackage{verbatim} %for code
\usepackage[pdftex]{graphics}
\usepackage{hyperref}
\usepackage{cleveref}
\usepackage{tikz}
\usepackage{comment}
\usepackage[nottoc]{tocbibind}
%\usepackage[square]{natbib}
\usepackage{caption}
\usepackage{subcaption}

\addtokomafont{disposition}{\rmfamily}
\usetikzlibrary{arrows,shapes}

%Graphis Extensions
\DeclareGraphicsExtensions{.png, .jpg}
\parindent 0pt

% Predefined things such as commands, etc.

\newcommand{\aRel}[1] {
  \sim_{#1} 
}

\newcommand{\kripkeFrame}[2] {
  (#1, \aRel{#2})
}

\newcommand{\kripkeModel}[3] {
  (#1, \aRel{#2}, #3)
}

\newcommand{\frKripModel}[2] { % defined via Kripke Frame + valuation
  (#1, #2)
}

\newcommand{\actModel}[3]{
  (#1, \aRel{#2}, #3)
}

\newcommand{\actModelStates}[4] {
	(#1, \aRel{#2}, #3, #4)
}

\newcommand{\frActModel}[2] { % defined via Kripke Frame + Pre
  (#1, #2)
}

\newcommand{\note}[1]{\textsc{\textbf{#1}}}
\newcommand{\Universal}{$\mathcal{U}$}
\newcommand{\modalLog}{$\mathcal{L}$}
\newcommand{\modLogInf}{$\mathcal{L}_\inf$}
\newcommand{\epActLog}{\modalLog$([\alpha])$}
\newcommand{\epActLogCommonKnowledge}{\modalLog$([\alpha],\box^{*})$}

\newtheorem{defn}{Definition}
\newtheorem{thm}{Theorem}
\newtheorem{lemma}{Lemma}
\newtheorem*{remrk}{Remark}

% Drawings of frames

\tikzstyle{vertex}=[circle,fill=black!25,minimum size=20pt,inner sep=0pt]
\tikzstyle{selected vertex} = [vertex, fill=red!24]
\tikzstyle{edge} = [draw,thick,->]
\tikzstyle{weight} = [font=\small]

%-----------------------------------------------------------------------------%
%Document%
\begin{document}

\section{Definitions}

\begin{defn} \label{frame}
	Let $\Sigma$ be a set and $\aRel{A}$ a set indexed by $A$, a set of agents.
	Then the tuple $\kripkeFrame{\Sigma}{A}$ is a frame.
\end{defn}

A frame represents a mathematical structure, but holds no meaningful information.
In order to give meaning and be able to draw conclusions from the frame, we can
construct a valuation function that transforms the frame into a model detailing
information.
\begin{defn} \label{model}
	Let $F = \kripkeFrame{\Sigma}{A}$ be a frame, and $A$ a set of agents.
	Let $V: \mathbb{P} \to \mathcal{P}(\Sigma)$ be a function mapping any
	well-formed	sentence in $\mathbb{P}$ to a subset of $\Sigma$.
	We say $V$ is a valuation function on $F$.\\
	\\
	Then $M = \frKripModel{F}{V} = \kripkeModel{\Sigma}{A}{V}$ is a Kripke
	model.
\end{defn}

Our models are now informative models that we can reason with.
In an epistemic logic context, we say that each $s \in \Sigma$ is a possible
world.
Each relation in $\aRel{A}$ is an accessibility relation, such that $s_1 \sim_a
s_2$ signifies an agent $a \in A$ considering that, if $s_1$ was the ``current"
world $a$ would believe $s_2$ was possible.\\
\\
We define an event model as a frame with a precondition function, and extend it
to take a set of 
{\em More stuff follows: some things about event models, etc.}\\
\\
Let us consider the following operation

\begin{defn} \label{gen_union_1}
	Let $M_1 = \actModelStates{\Sigma_1}{A}{pre_1}{\Gamma_1}$ and $M_2 =
	\actModelStates{\Sigma_2}{A}{pre_2}{\Gamma_2}$ be two event models with
	designated states.\\
	\\
	Let $M_1 \sqcup_B M_2 = \actModelStates{\Sigma}{A}{pre}{\Gamma}$ be the disjoint
	union of $M_1$ and $M_2$ for some group	of agents $B \subseteq A$, such that
	\begin{itemize}
		\item $\Sigma = \Sigma_1 \sqcup \Sigma_2$
		\item $\aRel{A} = {\aRel{A}}_1 \sqcup {\aRel{A}}_2 \sqcup \{w \sim_b w' | b \in
		B \land (w \in \Gamma_1 \land w' \in \Gamma_2 \lor w \in \Gamma_2 \land w' \in \Gamma_1)\}$
		\item $pre(w) = pre_1(w)$ if $w \in \Sigma_1$ and $pre(w) = pre_2(w)$ if $w
		\in \Sigma_2$
		\item $\Gamma = \Gamma_1 \sqcup \Gamma_2$
	\end{itemize}
\end{defn}

\begin{lemma}
	The algebra (?)
	\[
		\phi = \phi ; \theta | \phi ^{\ast} | \phi \sqcup \theta | \text{announce} |
		\text{skip} | \text{crash}
	\]
	generates all event models in {\bf K45}.
\end{lemma}

\begin{note}
	Actually this isn't true.
	As an example, you can generate a model outside of {\bf K45} and into ???.
\end{note}
\end{document}
