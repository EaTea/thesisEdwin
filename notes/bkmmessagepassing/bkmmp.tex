%-----------------------------------------------------------------------------%
%Packages%
\documentclass[10pt, a4paper, twoside]{article}
\usepackage{amsmath, amsfonts, listings, amssymb, mathtools, amsthm} %Mathematical Expressions package
\usepackage{mathtools}
\usepackage[usenames, dvipsnames]{color} %Color naming packages
\usepackage[margin=1.5cm]{geometry}
\usepackage{float}
\usepackage{verbatim} %for code
\usepackage[pdftex]{graphics}
\usepackage{ulem}
\usepackage{hyperref}
\usepackage{cleveref}
\usepackage{thmtools}
\usepackage{tikz}

\usetikzlibrary{arrows,shapes}

%Graphis Extensions
\DeclareGraphicsExtensions{.png, .jpg}
\parindent 0pt

% Predefined things such as commands, etc.

\newcommand{\aRel}[1] {
  \sim_{#1} 
}

\newcommand{\kripkeFrame}[2] {
  (#1, \aRel{#2})
}

\newcommand{\kripkeModel}[3] {
  (#1, \aRel{#2}, #3)
}

\newcommand{\frKripModel}[2] { % defined via Kripke Frame + valuation
  (#1, #2)
}

\newcommand{\actModel}[3]{
  (#1, \aRel{#2}, #3)
}

\newcommand{\frActModel}[2] { % defined via Kripke Frame + Pre
  (#1, #2)
}

\newcommand{\note}[1]{\textsc{\textbf{#1}}}
\newcommand{\Universal}{$\mathcal{U}$}

\newtheorem{defn}{Definition}
\newtheorem{thm}{Theorem}
\newtheorem{lemma}{Lemma}
\newtheorem*{remrk}{Remark}

% Drawings of frames

\tikzstyle{vertex}=[circle,fill=black!25,minimum size=20pt,inner sep=0pt]
\tikzstyle{selected vertex} = [vertex, fill=red!24]
\tikzstyle{edge} = [draw,thick,-]
\tikzstyle{weight} = [font=\small]

%-----------------------------------------------------------------------------%
%Document%
\begin{document}
Number of citations: 15\\
\\
The paper is written by Bollig, Kuske and Meinecke (BKM).
BKM find that if they specify sequences of message passing (MSC) extending LTC
and TLC.\\
\\
I can't find anything on LTL or TLC at the moment?
Maybe something to ask Tim.\\
\\
The major result of this paper is that a formula in PDL can be translated to a
communicating finite state machine of exponential size (number of states).
At least this is what's most interesting to me.\\
\\
BMS also discuss some interesting things with regards to model checking.
They show that PDL with intersection can't be effectively modelled by a CFM.
I think that they had a most interesting result in the first one I mentioend.
I should re-read this paper for a bit of guidance though.\\
\\
Major result is Corollary 3.8.
I reiterate it here.
\begin{thm}
	Let $\phi$ be a global formula of PDL.
	Then a CFM $\mathcal{A}$ that accepts $M$ iff $M \models \phi$.
	The numbers of local states and of control messages of $\mathcal{A}$ belong to
	$2^{O((|\phi|+|\mathcal{P}|)^2)}$.
\end{thm}

\end{document}
