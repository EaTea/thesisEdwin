%-----------------------------------------------------------------------------%
%Packages%
\documentclass[10pt, a4paper, twoside]{article}
\usepackage{amsmath, amsfonts, listings, amssymb, mathtools} %Mathematical Expressions package
\usepackage{mathtools}
\usepackage[usenames, dvipsnames]{color} %Color naming packages
\usepackage[margin=1.5cm]{geometry}
\usepackage{float}
\usepackage{verbatim} %for code
\usepackage[pdftex]{graphics}
\usepackage{ulem}
\usepackage{hyperref}

%Gra$\phi$cs Extensions
\DeclareGraphicsExtensions{.png, .jpg}
\parindent 0pt

%-----------------------------------------------------------------------------%
%Document%
\begin{document}
\begin{titlepage}
\begin{center}
\vspace*{10cm}
\Huge{Modal Logic: Chapter One} \\[1cm]
\LARGE{Blackburn, de Rijke and Venema} \\[1cm]
\large{Notes: {\bf Edwin Tay, 20529864}}\\[1cm]
\today
\end{center}
\end{titlepage}
\pagebreak

\section{Basic Modal Logic}
\begin{itemize}
  \item propositional modal logic is a propositional language which a modal operator has been added (modal operator === modality)
  \item useful for reasoning about relational structures
  \item modal languages are essentially ways of accessing information in relational structures
  \item local and internal access methods reason about many interesting/important aspects of relational structures
\end{itemize}
\subsection{Def 1.1 --- a relational structure}
A tuple $R$ consisting of a nonempty domain set, $W$ and a set of relations on $W$ (relational structures will be assumed to have relations)

\begin{itemize}
  \item Strict Partial Orders are transitive and irreflexive, whilst partial orders are the reflexive closure of a strict partial order
  \item A deterministic transition (state) model is a relational structure
\end{itemize}

\subsection{Def 1.6 --- Closures}
Let's say a closure is the smallest subset that has a given Mathematical Property and the original set
\begin{itemize}
  \item We say that the transitive closure of a set of relations $R$ is $R^+$ =   intersection of all transitive relations that contain $R$. It should be clear to see that this intersection yields $R$ plus the smallest amount of necessary transitive relations
  \item Similarly, we define the reflexive transitive closure of a set of relations $R$ to be $R*$ which is the intersection of all transitive and reflexive relations that contain $R$
\end{itemize}
\subsection{Def 1.7 --- Trees}
As defined normally, might be the tuple $(T, S)$
\begin{itemize}
  \item nodes, which we call $T$, and a root in $T$, namely $r$ such that $rSt$ always holds
  \item every element of $T$ that isn’t $r$ has a unique predecessor such that there's only one element $t'$ such that $t'St$
  \item $S$ is acyclic, that is $S$ is irreflexive
\end{itemize}

\subsection{Def 1.9 --- Basic modal language}
Is defined (in BNF) as
\[
\phi \Eqqcolon p | \bot | \neg \phi | \gamma v \phi | \lozenge \phi
\]

\begin{itemize}
  \item $\square$ is the dual of $\lozenge$
  \item $\square$ $\phi$ = $\neg$ $\lozenge$ $\neg$ $\phi$
  \item $\lozenge$ $\phi$ = $\neg$ $\square$ $\neg$ $\phi$
  \item we add in the following abbreviations
  \begin{itemize}
    \item and
    \item imply
    \item equivalence
    \item truth = $\neg$ $\bot$
  \end{itemize}
\end{itemize}
\begin{itemize}
  \item $\lozenge$ and $\square$ can be read in many d$\iff$erent ways
  \item $\lozenge$ $\phi$ can be read as 'it is possibly the case that $\phi$'
  \item $\square$ $\phi$ can be read as 'necessarily $\phi$'
  \item we can get interesting results like

  \item $\square$ $\phi$ $\implies$ $\lozenge$ $\phi$ (necessary $\implies$ possible)

  \item $\phi$ $\implies$ $\lozenge$ $\phi$ (to be implies possibility)
  \item but what about whatever is is necessarily possible? or whatever is possible is necessarily possible
  \item in epistemic logic $\square$ $\phi$ means that an agent knows $\phi$ (usually K $\phi$)
  \item similarly $\lozenge$ $\phi$ means an agent suspects $\phi$
\end{itemize}
\subsection{Def 1.11 --- A modal similarity type}
If a pair $(O, p)$ where $O$ is a nonempty set and $p$ is a function $O \rightarrow \mathbb{N}$. $O$'s elements are the modal operators, denoted by $\nabla$, $\nabla$, etc.
\begin{itemize}
  \item p assigns each $\nabla$ to a finite arity, indicating the number of arguments that can be applied to a $\nabla$
\end{itemize}
\subsection{Def 1.12 --- A modal language}
Is built up using a modal similarity and a set of proposition letters so that you can do things like
\begin{itemize}
  \item $\phi$ $\eqqcolon p$ | $bot$ | $\neg$ $\phi$ | $\phi_1$ $\lor$ $\phi_2$ | $\nabla_1$($\phi_1$, $\phi_2$, $\ldots$) | $\nabla_2$($\ldots$)| $\ldots$
  \item N.B.: It’s possible to get nullary modalities or modal constants (things
      that take no arguments)
  \item they aren’t very focused but they are useful, and are almost like
  propositional constants
\end{itemize}
\subsection{Def 1.13 --- dual operators}
for non-nullary $\nabla$s, so that the dual of $\nabla$ is the $\bigtriangleup$ and is defined as $\nabla_1$ ($\phi_1$, $\phi_2$, $\ldots$) = $\neg \bigtriangleup$($\neg$ $\phi_1$, $\neg$ $\phi_2$, $\ldots$)
\begin{itemize}
  \item Some discussions follow on basic temporal logic and other stuff:
  \item temporal logic has two operators
  \item $F$ and $P$
  \item $\phi$ will be true at some point in the Future
  \item $\phi$ was true at some point in the Past
  \item with duals $G$ and $H$
  \item $\phi$ was always Going to be true
  \item $\phi$ always Has been true
  \item in effect, we’re going to say that we express concrete unmovable facts with the duals, and transient facts with $F$ and $P$
  \item interesting results include
  \begin{itemize}
    \item $P$$\phi$ $\implies$ $G P$ $\phi$
    \item $F$ $\phi$ $\implies$ $F F$ $\phi$ (?): so there’s a time slice that’s in between that you can think about?!
    \item $G F$ $\phi$ $\implies$ $F G$ $\phi$: the McKinsey formula! (implies that if something is going to be true at some point in the future it makes it forever true!)
  \end{itemize}
  \item PDL
  \item $\lozenge$ === <$\pi$>$\phi$ := some execution of program $\pi$ yields $\phi$
  \item $\square$ === [$\pi$] $\phi$ := all executions of program $\pi$ yield $\phi$
  \item execution MUST terminate
  \item supposing we can choose which program to execute (union), we can compose
  programs ($\pi$1 ; $\pi$2) indicates execution of $\pi_1$ followed by execution of $\pi_2$
  and iteration to $\pi$* to execute a finite/possibly zero number of times
  \item this is regular PDL, but we can make it more powerful by adding in intersection, which yields parallel execution and test, which is an assert (test is denoted by a ? after a formula)
  \item There’s also arrow logic
  \item three operators, converse, identity and composition
  \item arrow logic is a bit odd?
  \item A Note: A partial function is a function that only maps a subset of its domain out to a range
\end{itemize}
\subsection{Def 1.19 --- A frame}
A pair $(W, R)$ such that $W$ is a non-empty set and $R$ is a binary relation on $W$
\begin{itemize}
  \item we say that a model is a tuple $(F, V)$ where $F$ is a frame and $V$ is a function that takes a proposition $\phi$ and maps it to a subset of $W$, so $V$: $\phi$ $\rightarrow$ $\mathbb{P}(W)$
  \item Notice that a frame is a structure, and a model is an enhanced frame, enriched by a set of unary relations
  \item frames and models are very d$\iff$erent
  \item the frames give you a picture that is interesting, and add precision to the fundamental assumptions
  \item models assign meaning to frames; they are a frame with a valuation, and dress our frames with contingent information
\end{itemize}
\subsection{Def 1.20 --- Satisfaction}
Supposing w is a state in a model $M = (W, R, V)$. We say a formula $\phi$ is satisfied or true in $M$ at state $w$ as follows:
\begin{itemize}
  \item $M, w$ $\Vdash$ $p \iff w \in V(p)$ where $p \in$ propositions of model
  \item $M, w \Vdash$ $\bot$ is never true
  \item $M, w \Vdash$ $\neg$ $\phi$ $\iff$ $\neg$ $M, w$ $\Vdash$ $\phi$
  \item $M, w \Vdash$ $\phi$ or $\gamma$ $\iff$ $M, w$ $\Vdash$ $\phi$ or $M, w$ $\Vdash$ $\gamma$
  \item $M, w \Vdash$ $\lozenge$ $\phi$ $\iff$ for some $v \in W$ and $vRw$ we have $M, v$ $\Vdash$ $\phi$
  \item $M, w \Vdash$ $\square$ $\phi$ $\iff$ for all $v \in W$ s.t. $wRv$ then $M, v$ $\Vdash$ $\phi$
  \item we say a set of formulas $\Pi$ are true at $M, w$ $\Vdash$ $\Pi$ $\iff$ all members of $\Pi$ are true at $w$
  \item N.B.: this notion of satisfaction is intrinsically local and internal; we evaluate formulas inside models at a state $w$ (the current state).
\end{itemize}
\begin{itemize}
  \item $\lozenge$ works locally, where its definition is to scan all accessible states to find a candidate good state
  \item We say $M, w$ ||/- $\phi$ means $\phi$ is $\neg$ satisfied in $M, w$
  \item Note that $V$ tends to be extended to evaluate formulae, rather than propositions (just makes it easier), so that $V$($\phi$) := {$w$ | $M, w$ $\Vdash$ $\phi$} where $\phi$ is some formula (notice that this will reduce to $V$($\phi$) when evaluating the formulae properly)
\end{itemize}

\subsection{Def 1.21 --- globally or universally true}
...in a model $M$ if it is satisfied at any $w$ of $M$
\begin{itemize}
  \item if it is satisfied in some state, it is satisfiable (there is some state where $\phi$ is true)
  \item if the negation is satisfiable then the formula is falsifiable
  \item Note: $\square$ is satisfied at dead ends
\end{itemize}
\subsection{Def 1.23 --- modal similarity type}
$T$ is a modal similarity type. a $T$-frame is a tuple with the following ingredients
\begin{itemize}
  \item a non-empty set $W$
  \item for each $n \geq 0$ and each $n$-ary modal operator $\nabla$ in the similarity type $T$ an $(n+1)$-ary relation $R_\nabla$
  \item notice that we can then extend our T-frames for an operator $\nabla$, such that $M, w$ $\Vdash$ $\nabla$($\phi_1$, $\phi_2$, $\ldots$) $\iff$ there are some states $v_1$, $v_2$, $\ldots, \in W$ with $R_\nabla w,v_1,v_2,\ldots$ such that for each $i$, $M,v_i$ $\Vdash$ $\phi$i
  \item similarly for $\square$, we say that $\nabla$($\phi_1$,$\phi_2$,...) is true at $M,w$ $\iff$ for all combinations of states $v_1,v_2,\ldots \in W$ such that $R_\nabla w,v_1,v_2,\ldots$ then for each $i$, $M,v_i$ $\Vdash$ $\phi_i$
  \item Often we’d like to ignore the effects of valuation and get a grip on the more fundmental level of frames
\end{itemize}
\subsection{Def 1.28 --- Validity}
A formula $\phi$ is valid at a state $w$ in a frame $F$ $F, w$ $\Vdash$ $\phi$) if $\phi$ is true at w in every model (F, V) based on F;
\begin{itemize}
  \item $\phi$ is valid in a frame $F$ if it is valid at every state in $F$
  \item $\phi$ is valid on a class of frames $F$ if it is valid on every frame $F$ in $F$
  \item $\phi$ is valid if it is valid on all classes of frames
  \item the set of formulas that are valid in a class of frames $F$ is called the logic of $F$
  \item this gives rise to specific logics, such as the logic of S5 or KD45 or K
  \item notice that this essentially means we can assess validity independent of valuation
  \item We can say that $L_F$ which is the logic of a class of frames $F $is the set of valid formulas that can be made with operators in the modal similarity $T$, such that {$\phi$ in Form($T$, $\phi$) | F $\Vdash$ $\phi$}
  \item So what’s the problems with what we have right now?
  \item it’s too concrete when we talk about satisfaction and a little lacking
  in power (it talks about a single frame only)
  \item it’s too abstract and therefore a little too strong? when we talk about validity it goes across every valuation
  \item Let’s have an intermediate level --- a “general frame”, which is a frame $F$ with a restricted but suitably well-behaved collection of admissible valuations, $A$ (which can also stand for a boolean algebra?)
\end{itemize}
\subsection{Def 1.30 --- Relation Function}
\begin{itemize}
  \item Let $R$ be an $n+1$-ary relation on a set $W$
  \item We define the following $n$ ary operation $m_R$ on the power set $\mathbb{P}(W)$ of $W$ $m_R(X1,...,Xn)$ = ${w \in W | Rww_1w_2w_3...w_n for some w_1 \in X1, …, w_n \in Xn}$
  \begin{itemize}
    \item $m_R(X)$ is the set of states that see a state in $X$, so these are the set of states that can be “seen” from all of X1, X2, ...
    \item that is, any $w$ is a state that is related by $R$ to a state in $X1$, a state in $X2$, etc.
    \item so this is the set of all R-related states to $X1$, $X2$, ...?
  \end{itemize}
\end{itemize}
\subsection{Def 1.32 --- General Frames}
Let $T$ be a modal similarity. A general $T$-frame is a pair $(F, A)$ where $F=(W,R_\nabla)$ is a $T$-Frame and $A$ is a non-empty collection of “admissible” subsets of $W$ closed under the following operations
\begin{itemize}
  \item union
  \item relative complement ($X \in A \implies W \ X \in A)$
  \item $X1,X2,...,Xn \in A$ $\implies$ $m_{R\nabla(X1,X2,...,Xn)} \in A \forall \nabla \in T$
  \item A model based on a general frame $(F, A, V)$ has a valuation such that $V(p) \in A$ for all propositions $p$
  \item Note that all the validities we mentioned in Def 1.28 hold for general frames
  \item So we’ve talked a lot about logically interesting formulas but we haven't discussed what a logical consequence might be
  \item indeed, we’ve said nothing about what it means for a set of modal formulas to logically entail another modal formula
\end{itemize}
\subsection{Def 1.35 --- Local Semantic Consequence}
\begin{itemize}
  \item Let $T$ be a similarity type and let S be a class of structure of type $T$ (a class of models, frames or general frames)
  \item Let $\sigma$ and $\phi$ be a set of formulas and a single formula from a language of type $T$
  \item We say that $\phi$ is a local semantic consequence of $\sigma$ over $S$ if for all models $M$ from $S$ and all points $w$ in $M$, if $M,w$ $\Vdash$ $\sigma$ $\implies$ $M, w$ $\Vdash$ $\phi$
\end{itemize}
\subsection{Def 1.37 --- Global Semantic Consequence}
\begin{itemize}
  \item We say that $\phi$ is a global semantic consequence of $\sigma$ over $S$ $\iff$ for all structures $G$ in $S$, if $G$ $\Vdash$ $\sigma$ then $G$ $\Vdash$ $\phi$
\end{itemize}
\subsection{Def 1.39 --- {\bf K}-proof}
\begin{itemize}
  \item A {\bf K}-proof is a finite sequence of formulas, each of which is an axiom or
  follows from one or more earlier items in the sequence by applying a rule of
  proof
  \item {\bf K} has the {\bf K} axiom: $\square$(p$\implies$q) $\implies$ ($\square$ p $\implies$ $\square$ q), and dual, that $\square$ p
  $\implies$ $\neg$ $\lozenge$ $\neg$ p
  \item {\bf K} has as rules of proof
  \begin{itemize}
    \item modus ponens
    \item uniform subsitution: given $\phi$, prove theta where theta is obtained from $\phi$ by uniformly replacing proposition letters in $\phi$ by arbitrary formulas
    \item generalisation: given $\phi$, prove $\square$ $\phi$
  \end{itemize}
\end{itemize}
\subsection{ Def 1.42 --- Normal Modal Logic}
\begin{itemize}
  \item A normal modal logic is a set of formulas that contains all tautologies,
  K, and dual and is closed under modus ponens, uniform substitution and
  generalisation
\end{itemize}

%-----------------------------------------------------------------------------%
\end{document}
