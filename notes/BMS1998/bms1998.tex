%-----------------------------------------------------------------------------%
%Packages%
\documentclass[10pt, a4paper, twoside]{article}
\usepackage{amsmath, amsfonts, listings, amssymb, mathtools, amsthm} %Mathematical Expressions package
\usepackage{mathtools}
\usepackage[usenames, dvipsnames]{color} %Color naming packages
\usepackage[margin=1.5cm]{geometry}
\usepackage{float}
\usepackage{verbatim} %for code
\usepackage[pdftex]{graphics}
\usepackage{ulem}
\usepackage{hyperref}
\usepackage{tikz}

\usetikzlibrary{arrows,shapes}

%Graphis Extensions
\DeclareGraphicsExtensions{.png, .jpg}
\parindent 0pt

% Predefined things such as commands, etc.

\newcommand{\aRel}[1] {
  \sim_{\mathcal{#1} }
}

\newcommand{\kripkeFrame}[2] {
  (#1, \aRel{#2})
}

\newcommand{\kripkeModel}[3] {
  (#1, \aRel{#2}, #3)
}

\newcommand{\frKripModel}[2] { % defined via Kripke Frame + valuation
  (#1, #2)
}

\newcommand{\actModel}[3]{
  (#1, \aRel{#2}, #3)
}

\newcommand{\frActModel}[2] { % defined via Kripke Frame + Pre
  (#1, #2)
}

\newcommand{\note}[1]{\textsc{\textbf{#1}}}
\newcommand{\Universal}{$\mathcal{U}$}
\newcommand{\modalLog}{$\mathcal{L}$}
\newcommand{\modLogInf}{$\mathcal{L}_{\infty}$}
\newcommand{\epActLog}{\modalLog $([\alpha])$}
\newcommand{\epActLogCommonKnowledge}{\modalLog$([\alpha],\Box^{\ast})$}

\newtheorem{defn}{Definition}
\newtheorem{thm}{Theorem}
\newtheorem{lemma}{Lemma}
\newtheorem*{remrk}{Remark}

% Drawings of frames

\tikzstyle{vertex}=[circle,fill=black!25,minimum size=20pt,inner sep=0pt]
\tikzstyle{selected vertex} = [vertex, fill=red!24]
\tikzstyle{edge} = [draw,thick,-]
\tikzstyle{weight} = [font=\small]

%-----------------------------------------------------------------------------%
%Document%
\begin{document}
Citations: 518\\
\\
Baltag, Moss and Solecki seem to define an idea similar to their future work in
Logic for Epistemic Programs.
In particular, they discuss a similar premise to their 2004 paper.
Given a state of inftyormation with deterministic propositions to decide upon, and
using a possible worlds model, how should the state of inftyormation be updated?\\
\\
BMS define a Logic of Epistemic Actions, \modalLog$([\alpha])$.
In a manner {\em very} similar to Baltag and Moss, 2004, this logic of epistemic
actions uses (for its action structure) a Kripke frame $K$ with a map of
preconditions from $K$ to sentences in \modalLog.\\
\\
In particular, BMS and define the interesting epistemic updates to be
announcements and suspicions and focus their efforts accordingly.
This paper seems to stop short of defining a logic for epistemic programs in a
vein similar to PDL, and instead looks at the completeness and expressivity of
\modalLog$([\alpha])$.\\
\\
One really interesting thing is the examples that come out of this paper.
BMS claim in section 3 that it is possible to represent the inftyormation content
of message passing on faulty or wiretapped channels using their action
framework.
This is really cool, but they don't provide specifics; merely that to the
external, omniscient third-party, the model of the world can be decided but that
the individual agents will not be able to effect these changes properly.
Why?
The main problem is that these agents are going to be looking at an entire
message passing situation as opposed to a pointed model.
Of note is that BMS define a pointed model execution to give ``current world
semantics", which might be more useful later on?
\begin{defn}
We say a sentence $\phi \in $\modalLog$([a])$ is a normal form iff it is a
modal sentence that contains no actions.\\
A sentence $\phi \in$ \modalLog$([a],\Box^{\ast})$ is a normal form iff it is built
from atomic propositions involving modal logic plus infinitary box, or it is of
the form $[\alpha]\Box^{\ast}\tau$ where $\alpha$ is an action in normal form and
$\tau$ is in normal form.\\
An action $\alpha$ is a normal form action if each $PRE(\beta)$ is a normal form
sentence for all $\beta$ such that $\alpha \implies^{\ast} \beta$.
\end{defn}
I think that definition is just giving us the ``basic" atomic sentences in
\modalLog$([\alpha])$ and \modalLog$([\alpha], \Box^{\ast})$.
In particular it gives us the idea of a basic action; one whose neighbours all
have normal form preconditions.
Note that this makes intuitive sense in S5 or KD45 but it does not generalise
intuively to K (because now you can have a basic action with non-basic
preconditions...)
\begin{defn}
We define \modLogInf to be infinitary modal logic, which are conjunctions (ands
or intersections) of arbitrary sets of sentences.
\end{defn}
\begin{thm}
Let \modalLog be modal logic, \modalLog$([\alpha])$ be modal logic with action
execution, \modalLog$([\alpha],\Box^{\ast})$ be \modalLog$([\alpha])$ with common
knowledge (infinitary box), and let \modLogInf be infinitary modal logic.\\
Then a (normal?) sentence in \modalLog$([\alpha])$ can be translated to a
sentence in \modalLog.\\
Furthermore, the translation used for \modalLog$([\alpha])$ to \modalLog can be
extended to translate \modalLog$([\alpha],\Box^{\ast})$ to \modLogInf.
This is achieved by making the translated sentence in \modLogInf to be a
recursive sentence. (of ordinal height $\omega^\omega$?).
\end{thm}
BMS also discuss the completeness of \epActLog.
Indeed they show that \epActLog is strongly complete.
\begin{thm}
The logical system \epActLog is strongly complete, that is
\[
  \Sigma \vdash \phi \iff \Sigma \models \phi
\]
\end{thm}
They also find a sentence in \epActLogCommonKnowledge that cannot be expressed
in \modLogInf.
This shows \epActLogCommonKnowledge is more expressive than \modLogInf.
They conclude that the validities of \epActLogCommonKnowledge can be axiomatised
along the lines of PDL.
As a result we get the finite model property and hence decidability.\\
\\
I think I ought to look at the final result and understand/ask Tim about it a
little more.
Also the differences in completeness results are worth exploring.
I think I need to look more into it.\\
\\
BMS suggest a number of questions they're interested in:
\begin{itemize}
  \item Communications and actions relations haven't been explored
  \item They publish some new results in BM2004 regarding their new program
  logic
  \item Completeness theorems for S4 and S5
  \item Incorporating ideas from belief revision into our framework
\end{itemize}

\end{document}
