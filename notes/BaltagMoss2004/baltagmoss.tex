%-----------------------------------------------------------------------------%
%Packages%
\documentclass[10pt, a4paper, twoside]{article}
\usepackage{amsmath, amsfonts, listings, amssymb, mathtools, amsthm} %Mathematical Expressions package
\usepackage{mathtools}
\usepackage[usenames, dvipsnames]{color} %Color naming packages
\usepackage[margin=1.5cm]{geometry}
\usepackage{float}
\usepackage{verbatim} %for code
\usepackage[pdftex]{graphics}
\usepackage{ulem}
\usepackage{hyperref}
\usepackage{tikz}

\usetikzlibrary{arrows,shapes}

%Graphis Extensions
\DeclareGraphicsExtensions{.png, .jpg}
\parindent 0pt

% Predefined things such as commands, etc.

\newcommand{\aRel}[1] {
  \sim_{\mathcal{#1} }
}

\newcommand{\kripkeFrame}[2] {
  (#1, \aRel{#2})
}

\newcommand{\kripkeModel}[3] {
  (#1, \aRel{#2}, #3)
}

\newcommand{\frKripModel}[2] { % defined via Kripke Frame + valuation
  (#1, #2)
}

\newcommand{\actModel}[3]{
  (#1, \aRel{#2}, #3)
}

\newcommand{\frActModel}[2] { % defined via Kripke Frame + Pre
  (#1, #2)
}

\newcommand{\note}[1]{\textsc{\textbf{#1}}}
\newcommand{\Universal}{$\mathcal{U}$}

\newtheorem{defn}{Definition}
\newtheorem{thm}{Theorem}
\newtheorem{lemma}{Lemma}
\newtheorem*{remrk}{Remark}

% Drawings of frames

\tikzstyle{vertex}=[circle,fill=black!25,minimum size=20pt,inner sep=0pt]
\tikzstyle{selected vertex} = [vertex, fill=red!24]
\tikzstyle{edge} = [draw,thick,-]
\tikzstyle{weight} = [font=\small]

%-----------------------------------------------------------------------------%
%Document%
\begin{document}
Citations: 279\\
\\
Baltag and Moss use an example-motivated approach to explore how they can
capture informative updates in situations regarding knowledge.
They formulate the following theses, without defense:
\begin{thm}\label{situationModel}
For any situation $s$ with deterministic beliefs and knowledge, we can model
that situation using a corresponding model $S$.
A statement about $s$ can be formalised to a formal statement about $S$.
Similarly, a formal statement about $S$ corresponds to a comment we can make
regarding $s$.
\end{thm}
Furthermore, they state that
\begin{thm}\label{actionModelExists}
Supposing an action, $\sigma$ operates on $s$ and changes it to a new situation
of information, $\sigma(s)$.
Then there exists an $\Sigma$ ``action model" that formally corresponds to
$\sigma$.
Statements about $\sigma$ can be formalised as formal statements about $\Sigma$
and vice versa.
Furthermore, for a situation $s$ and its corresponding mathematical model $S$ we
can execute $\Sigma$ on $S$ via operation $\otimes$ to retrieve a new Kripke
model $S \otimes Sigma$ which is faithful to the update $\sigma(s)$.
\end{thm}
\ref{situationModel} isn't very exciting, since it appears to offer the same
idea about modelling information as other papers.
It justifies the existence and usage of the Kripke model.
Conversely, \ref{actionModelExists} is really interesting, because there's now
a claim that you can construct these formal models for the actions themselves,
not just the situation.
This in itself is quite novel since modelling any class of epistemic updates
doesn't appear to have been done before.\\
\\
Baltag and Moss define an Action Model as follows:
\begin{defn}
Let $\Sigma$ be a set of actions that are ``simple" in their execution.
They have a ``uniform" appearance to agents, so that their impact at any current
state is the same.
We say that $\aRel{A}$ is a set of relations between sets in $\Sigma$ indexed
by a set of agents $\mathcal{A}$.
Let $pre$ be a function mapping an action to a sentence $\alpha$, a valid
logical sentence describing the preconditions for an action to be executed.
The tuple ${\bf \Sigma} = \actModel{\Sigma}{A}{pre}$ is thus an action model.
\end{defn}
\note{Notice that together, $\kripkeFrame{\Sigma}{A}$ is a frame in
itself and thus an action model is also a multi-agent Kripke model, but it's a
model of an epistemic update as oppposed to an epistmic situation.}\\
\\
Now, the authors introduce the idea of ``distinguished" actions that can
semantically correspond to the actions that are ``possible resolutions" of a
complex epistemic update.
This allows us to specify a program model that corresponds to an epistemic
update.
This means that the action model ${\bf \Sigma} = \actModel{\Sigma}{A}{pre}$ can
attain a higher level of specificty and thus the program model language could be
more specific?\\
\\
Baltag and Moss then define operations to compose and perform the disjoint union
of two program models (actually scales to multiple).
Furthermore they also show how to execute a program model on a situation $S$.\\
\\
We then have a concept of an ``action signature".
\begin{defn}
Let ${\bf \Sigma} = \actModel{\Sigma}{A}$ be a Kripke frame.
Let ${\bf \Phi} = \{\phi_1, \phi_2, \ldots \}$ where $\phi_i \subseteq \Sigma$.
We say that ${\bf \Gamma} = \frActModel{ {\bf \Sigma} }{ {\bf \Phi} }$ is an action
signature, where fixing one $\phi_i$ as the set of actions in ${\bf \Sigma}$ 
that have non-trivial (true) preconditions yields an action model.
\end{defn}
\note{Essentially, an action signauture is for action models what frames are
for Kripke models.}
\end{document}
