%-----------------------------------------------------------------------------%
%Packages%
\documentclass[10pt, a4paper, twoside]{article}
\usepackage{amsmath, amsfonts, listings, amssymb, mathtools, amsthm} %Mathematical Expressions package
\usepackage{mathtools}
\usepackage[usenames, dvipsnames]{color} %Color naming packages
\usepackage[margin=1.5cm]{geometry}
\usepackage{float}
\usepackage{verbatim} %for code
\usepackage[pdftex]{graphics}
\usepackage{ulem}
\usepackage{hyperref}
\usepackage{tikz}

\usetikzlibrary{(arrows,calc,intersections,shapes)}

%Graphis Extensions
\DeclareGraphicsExtensions{.png, .jpg}
\parindent 0pt

% Predefined things such as commands, etc.

\newcommand{\aRel}[1] {
  \sim_{#1} 
}

\newcommand{\kripkeFrame}[2] {
  (#1, aRel{#2})
}

\newcommand{\kripkeModel}[3] {
  (#1, aRel{#2}, #3)
}

\newcommand{\frKripModel}[2] { % defined via Kripke Frame + valuation
  (#1, #2)
}

\newcommand{\actModel}[3]{
  (#1, aRel{#2}, #3)
}

\newcommand{\frActModel}[2] { % defined via Kripke Frame + Pre
  (#1, #2)
}

\newcommand{\note}[1]{\textsc{\textbf{#1}}}
\newcommand{\Universal}{$\mathcal{U}$}

\newtheorem{defn}{Definition}
\newtheorem{thm}{Theorem}
\newtheorem{lemma}{Lemma}
\newtheorem*{remrk}{Remark}

% Drawings of frames

\tikzstyle{vertex}=[circle,fill=black!25,minimum size=20pt,inner sep=0pt]
\tikzstyle{selected vertex} = [vertex, fill=red!24]
\tikzstyle{edge} = [draw,thick,-]
\tikzstyle{weight} = [font=\small]

%-----------------------------------------------------------------------------%
%Document%
\begin{document}
Baltag and Moss use an example-motivated approach to explore how they can
capture informative updates in situations regarding knowledge.
They formulate the following theses, without defense:
\begin{thm}\label{situationModel}
For any situation $s$ with deterministic beliefs and knowledge, we can model
that situation using a corresponding model $S$.
A statement about $s$ can be formalised to a formal statement about $S$.
Similarly, a formal statement about $S$ corresponds to a comment we can make
regarding $s$.
\end{thm}
Furthermore, they state that
\begin{thm}\label{actionModelExists}
Supposing an action, $\sigma$ operates on $s$ and changes it to a new situation
of information, $\sigma(s)$.
Then there exists an $\Sigma$ ``action model" that formally corresponds to
$\sigma$.
Statements about $\sigma$ can be formalised as formal statements about $\Sigma$
and vice versa.
Furthermore, for a situation $s$ and its corresponding mathematical model $S$ we
can execute $\Sigma$ on $S$ via operation $\otimes$ to retrieve a new Kripke
model $S \otimes Sigma$ which is faithful to the update $\sigma(s)$.
\end{thm}
\cref{situationModel} isn't very exciting, since it appears to offer the same
idea about modelling information as other papers.
It justifies the existence and usage of the Kripke model.
Conversely, \cref{actionModelExists} is really interesting, because there's now
a claim that you can construct these formal models for the actions themselves,
not just the situation.
This in itself is quite novel since modelling any class of epistemic updates
doesn't appear to have been done before.\\
\\
\end{document}
