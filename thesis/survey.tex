\chapter{Literature survey} \label{lit_survey}

Within this chapter we consider the relevant writings and logics to describe
the epistemic state and how we can change and informatively update the epistemic
state.
We will explore how we can model static situations and
different informative updates.
We will note what has not been explored in the fields and compare frameworks'
differing approaches and strengths.\\
\\
To explore the techniques used to model information and the change of information, we introduce an
example of a game of information.
Consider a game being played between two friends (agents), Angeline $(A)$
and Ben $(B)$.
The game involves flipping a coin, hiding the result from both $A$ and $B$ and
having both of them guess whether the coin is Heads $(H)$ or the coin is Tails
$(T)$.
The game is refereed by a mutually trusted (and usually impartial) friend, Carol
$(C)$ who knows if the coin is $H$ or $T$.\\
\\
$A$ and $B$ know that either $H$ is true or $T$ is true.
They also know that $H$ and $T$ cannot both occur simultaneously together.
An agent's knowledge (which is inclusive of, but not limited to the previous
statements) is called the epistemic state.
Furthermore, there are two possibilities for the coin's outcome: that it is $H$
or $T$.
Neither $A$ or $B$ as they are can distinguish between them; that is, they are
uncertain which of the outcomes is true.
Agent uncertainty in this multi-agent system is another aspect we must attempt
to capture.

\section{Epistemic modal logic}\label{survey_epistemic_modal_logic}
In our coin-flipping game, we can make the following observations.
\begin{itemize}
	\item $A$ considers $H$ to be possible, as well as $T$ to be possible
	\item $B$ also considers $H$ to be possible, and also considers $T$ to be possible
\end{itemize}
We claim that there are two possible worlds that differ in one way ---
in one of these worlds, $H$ is true, and in the other world $T$ is true.
Furthermore, these worlds are the same from $A$ and $B$'s perspectives.
$A$ would not be able to distinguish the world where $H$ was true from either of
the world where $H$ was true or the world where $T$ was true, since she cannot
see the coin.\\
\\
We can represent this uncertainty between what world is true as relations
between worlds.
In this case, let $W = \{ \eta, \tau\}$ be our set of possible worlds, where $H$ is true
at $\eta$ and $T$ is true at $\tau$.
Let $R_A = R_B = \{(\tau,\tau), (\tau,\eta), (\eta, \tau), (\eta,\eta)\}$ be
binary relations on $W$ representing indistinguishability between two possible
worlds.
We call these relations ``accessibility relations"; we say if one world is
indistinguishable from another, then they can access each other.
Lastly, let $V$ be a valuation function that maps a formula to the set of worlds
where that formula is true, so $V(H) = \{\eta\}$ and $V(T) = \{\tau\}$.\\
\\

\begin{defn}
	The tuple
	\[
		M = (W, R = R_A \cup R_B, V)
	\]
	is an epistemic model of our game of heads and tails.
\end{defn}

These definitions, as well as a more formal, in-depth treatment of modal logic and
the possible worlds model, is given in reference texts by Blackburn, de Rijke and Venema \cite{blackburn2002modal},
as well as van Ditmarsch, van der Hoek and Kooi \cite{hoek2008dynamic}.
We also redefine models formally in Definition \ref{model}.

We can represent this model graphically as shown in Figure \ref{htkripkefigure},
where our possible worlds $(W)$ are nodes and our binary relations $(R$) are edges in the
graph.

\begin{figure}[ht!]
\centering
\begin{tikzpicture}[->,>=stealth',shorten >=1pt,auto,node distance=2cm,
      thick]

    \node[vertex] (1) {$\eta$};
    \node[vertex] (2) [right of=1] {$\tau$};
    \path[edge]
          (1) edge node {$A,B$} (2)
              edge [loop left] node {$A,B$} (1)
          (2) edge node {} (1)
              edge [loop right] node {$A,B$} (2);
\end{tikzpicture}
\caption{A graph representation of our game of heads and tails.}\label{htkripkefigure}
\end{figure}

The possible worlds models form a semantics allowing us to construct meaningful
models that reflect an epistemic state.
It is the semantics and models that allow us to reason about relations between the possible worlds in
an internal sense, from $A$ and $B$'s perspectives \cite{blackburn2002modal}.
We can define an operator to formally represent the modality of ``knowing"
a proposition or formula to be true.

\begin{defn}
	The operator $\Box_A H$ stands for ``$A$ knows $H$ is true".
	$\Box_A \phi$ is true at a possible world $w$ if for all accessibility relations $(w,
	w')$, $\phi$ is true.
\end{defn}

We also define the dual operator of knowing, which is to consider something
possible.

\begin{defn}
	The operator $\Diamond_B T$ stands for ``$B$ considers $T$ to be possible".
	$\Diamond_B \phi$ is true at a possible world $w$ if there is one
	accessibility relation $(w,w')$ such that $\phi$ is true at $w'$.
\end{defn}

Notice that we can define $\Diamond$ in terms of $\Box$, where $\Diamond
\phi \iff \neg \Box \neg \phi$.\\
\\
The definitions for $\Box$ and $\Diamond$ are given a more formal treatment in
Fagan et. al in \cite{fagin1995reasoning} and van der Hoek, Kooi and Ditsmarsch in
\cite{hoek2008dynamic}.
Both definitions are formally re-defined in Definition \ref{modalLogic}.\\
\\
We will omit the name of the agent for operators $\Box$ and $\Diamond$ when we
discuss a single-agent case, as follows.
We now show, without proof, four axioms that must hold for our modality of
knowledge.

\begin{propn}
	The following are true at any world of an epistemic model
	\cite{hoek2008dynamic}.
	\begin{enumerate}
		\item $\Box (\phi \implies \theta) \implies (\Box \phi \implies \Box
				\theta)$, known as {\bf K}
		\item $\Box \phi \implies \phi$, known as {\bf T}
		\item $\Box \phi \implies \Box \Box \phi$, known as {\bf 4}
		\item $\neg \Box \phi \implies \Box (\neg \Box \phi))$, known as {\bf 5}
	\end{enumerate}
\end{propn}

In order to satisfy these conditions, binary relations between possible worlds
in our epistemic models become equivalence relations.\\
\\
Also, if we replaced {\bf T} with an axiom {\bf D} which states $\Box \phi
\implies \Diamond \phi$, then we can model doxastic models of belief, instead of
epistemic models of knowledge.\\
\\
We can now make formal reasonings, such as deciding that $\Box_A H$ and $\Box_A
T$ are both false.
We note that $\Box_B (H \lor T)$ is true and $\Box_A \Box_B (H \lor T)$ is
true.\\
\\
{\bf K}, {\bf T}, {\bf 4} and {\bf 5} allow us to model epistemic state and
uncertainty in our game of heads and tails.
We have defined on our models that allows us to make formal reasonings and
decide if formulae are true at possible worlds.\\
\\
But what if their friend, $C$, announces ``The coin is Heads up"?
As they are, our epistemic models cannot describe the change in knowledge that
$C$'s announcement will entail.
Furthermore, we have no formal process to say how the situation has changed.\\
\\
This deficiency raises the following questions:
\begin{itemize}
	\item How do we describe this change in a formal manner?
	\item What reasoning can we make about the state of information after this
	change?
	\item Is there an operation that allows us to change the state of information
	from the pre-change state to the post-change state?
\end{itemize}

\section{Public announcement logic}\label{pal}
To model the announcement of facts and information, we turn to the use of public
announcement logic.
Public announcements are announcements to multiple agents of facts.
They are a kind of informative update --- perhaps the most basic kind.\\
\\
Public announcement logic was first proposed independently by both Plaza and
Gerbrandy and Groeneveld \cite{plaza2007public,gelbrandy1997reasoning}.
They were later augmented by the work of Baltag, Moss and Solecki through the
addition of common knowledge \cite{baltag1998lpa}.
Public announcements are informational updates of true facts that change the
knowledge state amongst multiple agents.\\
\\
In the context of our card game, let us consider if our external trusted party
$C$ publicly announces to $A$ and $B$ that $H$ is true.
We will demonstrate the execution of that update upon our model in Figure
\ref{pakripkefigure}.
\begin{figure}[ht!]
\centering
\begin{subfigure}[b]{.45\textwidth}
\centering
\begin{tikzpicture}[->,>=stealth',shorten >=1pt,auto,node distance=2cm,
      thick]

    \node[vertex] (1) {$\eta$};
    \node[vertex] (2) [right of=1] {$\tau$};
    \path[edge]
          (1) edge node {$A,B$} (2)
              edge [loop left] node {$A,B$} (1)
          (2) edge node {} (1)
              edge [loop right] node {$A,B$} (2);
\end{tikzpicture}
\caption{Our model before $H$ is announced to be true.}
\label{beforefigure}
\end{subfigure}
~
\begin{subfigure}[b]{.45\textwidth}
\centering
\begin{tikzpicture}[->,>=stealth',shorten >=1pt,auto,node distance=2cm,
      thick]

    \node[vertex] (1) {$\eta$};
    \path[edge]
          (1) edge [loop left] node {$A,B$} (1);
\end{tikzpicture}
\caption{Our model after $H$ is announced to be true.}
\label{afterfigure}
\end{subfigure}
\caption{The differences before and after the public announcements of $H$ to $A$ and
	$B$.}
\label{pakripkefigure}
\end{figure}
\\
Public announcements capture changes in information, such as $C$ saying to
$A$ and $B$ that the coin is $H$.
Another possible update is $A$ being allowed to see the coin's state, and
telling $B$ ``I guess you didn't know that the coin is actually heads up".\\
\\
These changes can be successful or unsuccessful, depending on whether the fact
that was announced is true after its announcement.
For example, if $A$ makes the (truthful) announcement that $H$, then this
announcement is successful since $H$ will be true after the announcement.\\
\\
Conversely, an announcement from $A$ to $B$ that ``I know that you {\em don't} know $H$
is true, but the coin is actually $H$".
We can write this as the announcement of the formula $H \land \neg \Box_B H$.
We can again refer to Figure \ref{pakripkefigure} to graphically show the models
before (Figure \ref{beforefigure}) and after (Figure \ref{afterfigure}) the update.
This update is interesting in that its announcement causes itself to be false.\\
\\
When $A$ publicly announces that the coin is $H$ and that $B$ does not know that
the coin is $H$ her announcement is now false, since $B$ now knows that $H$
actually is true.
Examining Figure \ref{afterfigure}, we can see that $H \land \neg \Box_B H$ is
untrue after its announcement in Figure \ref{beforefigure}.
This is an example of an unsuccessful public announcement.\\
\\
By using public announcement logic we can describe these announcements of
information.
We have a framework to describe and change the state of information amongst our
agents before and after announcements of facts.
Public announcement logic will allow us to model $C$ telling $A$ and $B$ in a
public fashion that $H$ is true.
This scales to modelling broadcasts in multi-agent systems, and allowing us to
update our models of the epistemic state of agents in such mass, public communications.\\
\\
In terms of dynamic epistemic logic, however, there could be many more kinds of
updates that we have yet to consider.
Re-examining our game of heads and tails yields scenarios that we cannot
describe:
\begin{itemize}
	\item What about $A$ cheating and learning $H$ or $T$ without $B$'s knowledge?
	\item What if $C$ was to whisper (in front of $B$) whether the coin was $H$
	or $T$?
\end{itemize}
These are things public announcement logic cannot describe, and our inability to
model them raises more questions:
\begin{itemize}
	\item what other kinds of epistemic (or informative) updates exist?
	\item how can we describe other epistemic updates in a sensible manner?
	\item what kind of an execution is required for a more nontrivial update?
\end{itemize}

\section{Epistemic actions and action models} \label{estAct}
We can now successfully capture the simple act of announcing a fact.
However, in public announcements, we cannot capture certain updates to the
epistemic state, such as
\begin{itemize} 
  \item $C$ whispers to $A$ that $H$ is true and $B$ sees her whisper
  \item $C$ whispers to $A$ that $H$ is true without $B$ seeing
  \item $B$ suspects $A$ of cheating, but he isn't sure if she's cheated
\end{itemize}
Let us aim to generalise the ideas behind public announcements to capture other
informative updates.\\
\\
We will now constrain our agents' behaviours to only being able to accept facts.
They will not worry about changing facts, and if facts do change we will take it
as something that causes our agents knowledge systems to simply crash.\\
\\
We will examine two contrasting approaches to model these epistemic updates.
Our investigation will show the main differences in terms of what the frameworks
can describe and how we can use them to reason about updates.
\subsection{Epistemic Relational Actions} \label{epi_acts}
van Ditmarsch approaches epistemic actions from a syntactical point of view,
aiming to create a language to specify actions in.
His work in constructing an epistemic action syntax extends some of the
syntax of established languages such as propositional dynamic
logic \cite{ditmarsch99knowledge,ditmarsch2002dga}.\\
\\
van Ditmarsch provides operations to construct dynamic epistemic formulae, and
further extends the language with dynamic or action-oriented constructs.
These include the ability to test a proposition, to update a group of agents'
knowledge and to make a non-deterministic choice between actions.
Thus, van Ditmarsch presents a way to describe complex actions, and indeed we can
express and describe all the actions we've discussed in the previous sections.
We can formally describe actions such as cheating, or a private
announcement to $B$ that $H$ is true that's seen by $A$.\\
\\
However, reasoning about epistemic actions in their current form is difficult.
Indeed, the interpretations of an epistemic action are non-trivial to
understand.
Moreover, van Ditmarsch, van der Hoek and Kooi present problems with
representing complex uncertainties with van Ditmarsch's epistemic actions that make it difficult
to reason about what $A$ and $B$ know after an action takes place.\\
\\
In the next section, we review a different framework to model epistemic updates
which can describe as many updates as van Ditmarsch's language.
This framework resolves the issues with van Ditmarsch's epistemic actions and
has uncertainty about actions naturally encoded within it.
\subsection{Epistemic action models} \label{act_mods}
As an alternative to approaching epistemic actions syntactically, Baltag, Moss
and Solecki examine epistemic actions from a modelling point of view.
They aim to construct {\em epistemic action models}, which resemble the possible 
worlds models that were discussed in Section \ref{survey_epistemic_modal_logic}
\cite{baltag1998lpa}.
Aucher resolves this into a possible actions model that allows us to make
reasonings about updates that are similar to reasonings about the epistemic
state \cite{aucher09revisited}.
This ``possible actions" model addresses the issues regarding reasoning about
actions and their outcomes using van Ditmarsch's framework, as discussed in
Section \ref{epi_acts} \cite{hoek2008dynamic}.\\
\\
We will consider the following situation, where our game of heads and tails
begins with both $A$ and $B$ unable to discern if the coin is $H$ or $T$ (Figure
\ref{ambeforefigure}).
However, $B$ leaves the room to use the bathroom and when he returns he is unable
to tell if $A$ has looked at the coin and learned that it is $H$, $T$ or whether
$A$ has not looked and nothing has changed \ref{amafterfigure}.
The graphical models for each situation are presented in Figure
\ref{amkripkefigure}.\\

\begin{figure}[ht!]
\centering
\begin{subfigure}[b]{.45\textwidth}
\centering
\begin{tikzpicture}[->,>=stealth',shorten >=1pt,auto,node distance=2cm,
      thick]

    \node[vertex] (1) {$\eta$};
    \node[vertex] (2) [right of=1] {$\tau$};
    \path[edge]
          (1) edge node {$A,B$} (2)
              edge [loop left] node {$A,B$} (1)
          (2) edge node {} (1)
              edge [loop right] node {$A,B$} (2);
\end{tikzpicture}
\caption{Our model before $B$ temporarily left the game.}
\label{ambeforefigure}
\end{subfigure}
~
\begin{subfigure}[b]{.45\textwidth}
\centering
\begin{tikzpicture}[->,>=stealth',shorten >=1pt,auto,node distance=3cm,
      thick]

    \node[vertex] (1) {$\eta$};
    \node[vertex] (2) [right of=1] {$\tau$};
    \node[vertex] (3) [below of=1] {$\eta'$};
    \node[vertex] (4) [right of=3] {$\tau'$};
    \path[edge]
          (1) edge node {$A,B$} (2)
              edge [loop left] node {$A,B$} (1)
							edge node {$B$} (3)
							edge node {} (4)
          (2) edge node {} (1)
              edge [loop right] node {$A,B$} (2)
							edge node [above,pos = 0.33] {$B$} (3)
							edge node {$B$} (4)
					(3) edge node {} (1)
							edge node {} (2)
							edge node {$B$} (4)
							edge [loop left] node {$A,B$} (3)
					(4) edge node [above, pos = 0.66] {$B$} (1)
							edge node {} (2)
							edge node {} (3)
							edge [loop right] node {$A,B$} (4);
\end{tikzpicture}
\caption{Our model after $B$ returns and is unsure about whether $A$ has looked
	at the coin.
$\tau$ and $\eta$ are worlds where $A$ has not cheated while $\tau'$ and $\eta'$
are worlds where $A$ has cheated and looked at the coin.}
\label{amafterfigure}
\end{subfigure}
\caption{The differences before and after $B$'s temporary departure.}
\label{amkripkefigure}
\end{figure}

In Figure \ref{amkripkefigure}, we can say that $H$ is true at $\{\eta,\eta'\}$ and $T$ is
true at $\{\tau,\tau'\}$, but that in $\eta'$ and $\tau'$ $A$ has taken a look
at the coin and can distinguish between $\eta'$ and $\tau'$.
Our change in epistemic state is a change that we cannot model with public
announcement logic.
It is a private announcement such that $B$ is aware of the announcement, but not
of its contents.\\
\\
This update is one we can capture with an action model.
$B$ considers it possible that an action where $A$ looks at the coin and learns
$H$ occurs.
He also considers the action where $A$ looks at the coin and learns $T$ possible
as well.
Lastly, he considers that $A$ did not look at the coin and thus the state of the
world remains the same, signified by {\bf tr}.
As $B$ cannot tell which one of them occurred, but is aware that one of them did
occur, he considers them indistinguishable.
We can represent them as a ``possible actions" model, as we show in Figure
\ref{amprivatea}.
Baltag, Moss and Solecki define an operation to execute the model in Figure
\ref{amprivatea} on Figure \ref{ambeforefigure}, yielding Figure
\ref{amafterfigure} \cite{baltag1998lpa}.

\begin{figure}[H]
\centering
\begin{tikzpicture}[->,>=stealth',shorten >=1pt,auto,node distance=3cm,
      thick]

    \node[vertex] (1) {$H$};
    \node[vertex] (2) [right of=1] {$T$};
    \node[vertex] (3) [below of=1] {{\bf tr}};
    \path[edge]
          (1) edge node {$B$} (2)
							edge node {$B$} (3)
              edge [loop left] node {$A,B$} (1)
          (2) edge node {} (1)
							edge node {$B$} (3)
              edge [loop right] node {$A,B$} (2)
					(3) edge [loop below] node {$A,B$} (3)
							edge node {} (1)
							edge node {} (2);
\end{tikzpicture}
\caption{An example of an action model; in this example, the update models $B$
becoming suspicious of $A$ having learnt whether $H$ or $T$ is true, but not
knowing whether the $A$ actually has not learnt anything and the world remains
as it was before (signified {\bf tr}).}
\label{amprivatea}
\end{figure}

Baltag and Moss extend the action models they proposed with Solecki into classes
of action models, which they term action signatures \cite{baltag2005programs}.
They formalise general descriptions of actions, such as announcements, private
announcements, lying, suspicion and other epistemic updates.
As an example, they generalise all public announcements to a single structure,
showing the superior expressiveness of action models.
They also specify a syntax for discussing actions of a given action signature,
allowing them to generate the logic of public announcements in its entirety.
Within this syntax, the action model structures are used as syntactical objects.
Like van Ditmarsch's epistemic actions, Baltag and Moss' syntax extends previous
logics of dynamics and change.\\
\\
Action models are define a method of execution upon an epistemic model that is well-understood.
The execution will update a model of knowledge from a pre-update to a
post-update state.
Unfortunately, the cost of action model execution (that is, to transform a state
of information to a new state using an update specified by an action model) is
quite high.
It involves taking a Cartesian product of all possible actions.
This indicates that, given $N$ possible actions in an action model, $N^2$
actions must be computed in the new state of information before the model can
be refined.

\subsection{Comparison to relational epistemic actions} \label{epi_compare}
Both relational epistemic actions \cite{hoek2008dynamic} and action models
\cite{baltag1998lpa} describe a larger range of epistemic updates for our game of heads and tails.
The examples we chose earlier, regarding $C$ telling $A$ about the state of the
coin, or $B$ becoming suspicious of $A$ can be expressed using either framework.
What, then, is the real difference between employing action models, as opposed
to relational epistemic actions?\\
\\
van Ditmarsch, van der Hoek and Kooi \cite{hoek2008dynamic}, as well as Baltag and Moss
\cite{baltag2005programs} independently
note that the ``possible actions" of action models  are actually relational
epistemic actions.
In their own review of action models, van Ditmarsch, van der Hoek and Kooi
motivate this with an example.\\
\\
They consider the non-deterministic choice of 3 possible actions which are
indistinguishable externally.
As an example, $B$ might learn that $A$ knows either of $H$ or $T$, or has
learnt nothing at all.
Since $B$ does not know which of these actions could occur, each of these
epistemic updates is a possible action.\\
\\
They show that it is possible to express this uncertainty in a manner quite similar
to an action model.
Similarly, the corresponding action model of whether $B$ learns of $A$'s
knowledge of $H$, $T$ or nothing new is quite easily interpreted in a manner
akin to relational epistemic actions.
Their general result is that relational actions can be expressed as action
models, and vice versa.
As van Ditmarsch, van der Hoek and Kooi have already shown, it is in reasoning
with these models that the models differ.\\
\\
Notably, relational epistemic actions and action models suffer from the same
weakness, in that they are only externally describe an update of information.
Their execution and specification are only useful to a third-party viewer, and
how agents internal to a system update their knowledge is not clear.
This is one of several weaknesses in current dynamic epistemic logic, with
regards to translating a specification of an informational update into an
implementable series of messages.\\
\\
It is in their reasoning that action models and relational epistemic actions
differ most.
action models give us a notion of uncertainty between actions in a manner
similar to the possible worlds model, allowing us to use a formally established framework
to reason about dynamic updates.
This allows us to reason about our updates and their effects in a more natural
and well-understood way.\\
\\
Conversely, van Ditmarsch's relational epistemic actions give us a syntax that
is perhaps more natural than the one Baltag and Moss define.
Baltag and Moss are note that it is non-standard to use the modelling structures
inside a syntax \cite{baltag2005programs}.
In the context of interpreting and making sensible updates, however, it is
perhaps less useful to have abstraction at the cost of useful interpretations,
suggesting that action models are a more interesting way to model epistemic
updates.
Indeed, it would appear that at the moment action models are experiencing more
interest compared to relational epistemic actions, as we explore in the next
section.

\section{Extensions of action models}
Action models, when first introduced in \cite{baltag1998lpa}, were a concept that could be used to make
formal reasonings about actions in a similar way to possible worlds semantics.
Progress in this area has added a syntax that uses action models as syntactic
objects, improved the expressiveness of action models and investigated the
synthesis of action models.
Much of the discussion has showed how powerful action models are as a way for a
third party to reason about an informational update, and improved what kind of
informational updates we can capture.\\
\\
Baltag and Moss \cite{baltag2005programs} define a powerful syntax for constructing epistemic programs.
Their syntax is as powerful as van Ditmarsch's in relational epistemic actions,
being able to express sequences of epistemic actions or make a non-deterministic
choice between two established action models.
In our game, we would be able to model sequences of actions, such as $A$ finding
out whether $H$ is true or $T$ is true while $B$ watches, then $B$ finding out
whether $H$ or $T$ is true but without $A$ being aware of $B$ learning this.\\
\\
The only objection to their syntax, as raised earlier, is the curious use of
the semantic structures within a syntax.
It is also notable that the operations of composition and non-deterministic
choice can be as costly as action model execution, generating $N \times M$
possible actions, given two action models of size $N$ and $M$
respectively \cite{baltag2005programs}.\\
\\
van Benthem, van Eijck and Kooi \cite{benthem2006lcc} extend action models for a new language to
handle communication and change.
Their main contribution here is to improve action model languages with notions
of group knowledge and iterations of an action model's execution.
This logic of communication and change uses previous logics of change as a basis
and proceeds to interpret it in an epistemic fashion.\\
\\
van Ditmarsch, French and Pinchinat \cite{van2009simulation,van2010future} construct a future action model logic in
their contribution.
Together, their work shows how one can determine if for a particular possible
world it is possible to update it such that a condition $\phi$ is true after an
update.\\
\\
In a recent contribution, Hales \cite{hales13synthesis} describes an algorithm that can synthesise arbitrary
action models.
Hales builds upon van Ditmarsch and French's work, showing that if an action
model $\alpha$ exists to ensure a condition $\phi$ was true after that models
execution, then there is a method by which an equivalent action model can be generated.
This contribution is especially notable due to it generalising outside of
epistemic logic, to any modal logic system.\\
\\
This shows that we can use action models to ensure a condition is true
in a new epistemic state, if it is at all possible.
As an example, if in our game of coins $A$ and $B$ are unsure whether $H$ or $T$
are true, we would be able to determine if an action model exists such that
after its execution, $\Box_A H \land \Box_B H$ was true at a
specific world.
Employing the work of Hales' would allow us to synthesise this action model.\\
\\
An alternative approach is shown by Aucher in
\cite{doi:10.3166/jancl.21.289-321,doi:10.1080/11663081.2012.736703}, which is related to
to Hales' work.
Aucher discusses the concepts of epistemic planning.
To define epistemic planning, suppose that we have an information state where
some epistemic sentence $\phi$ is true.
We want to construct an information state such that the epistemic sentence
$\phi''$ is true.
Epistemic planning involves finding an update that moves from $\phi$ to
$\phi''$.
It is very similar to Hales' work, but unlike Hales' action models it allows a
finer degree of control about the situation prior to an event.\\
\\
Aucher provides a framework to infer properties about an update that will
realise $\phi''$ from $\phi$.
What is most interesting is that Aucher provides a framework to reason about the
properties of an update that realises $\phi''$ from $\phi$.
His contribution is novel, but parallel to the work of Hales' since it describes properties of event
models that can affect a change, whilst Hales describes the actual change itself and gives
rudimentary analysis of his procedure.\\
\\
The state of the art in action models and this section of dynamic epistemic
logic is thus focused on improving descriptions of information updates, or how
we can use them to achieve a given epistemic state.

\section{Belief revision}

Suppose that $A$ and $B$ are playing the heads-or-tails game with $C$ again.
$C$ checks the coin and announces ``The coin is heads up".
$A$ and $B$ now believe that the coin $C$ is holding is heads up.
But $C$ rechecks the coin and realises it is actually {\em tails}.
She apologetically announces to $A$ and $B$ that the coin is actually tails.\\
\\
Our current update frameworks cannot deal with the world being inconsistent with
our knowledge.
The moment that the beliefs and knowledge of an agent must be reconciled with
the world, agents cannot update their knowledge appropriately.
They stop being able to continue the state of the world.\\
\\
We will round out our literature survey by briefly addressing the field of
dynamic epistemic logic that allows us to revise the epistemic state of the
world.
This field is ``belief revision", where what agents believe to be true can be
revised to match the new state of the world.
Belief revision was first properly introduced in \cite{theLogicOfTheoryChange} and also
expanded on in \cite{gairdenfors1988knowledge}.
These papers built upon initial frameworks, such as the one Harper discusses in
\cite{harper1976rational}, or the truth maintenance systems for database
consistency which Doyle introduces in \cite{Doyle1979231}.\\
\\
Belief revision is more robust than the frameworks we considered as it allows
for beliefs that are inconsistent with the actual state of the world.
This robustness invites more obvious applications for revision belief in fields such as
artificial intelligence or database management and consistency.
It is a powerful approach in resolving conflicts between knowledge and facts.
This approach is tangential to what we are interested in, however, as we are
examining methods of constructing updates, as opposed to reconciling what we
know and what we have observed.

\section{Refinement quantified modal logics} \label{section:refineModalLogics}

A final, tangential approach to dynamic epistemic logic is to consider the concept of refinements.
Refinements are an alternate treatment of updates.
They are another way of viewing updates; van Ditmarsch and French show in \cite{van2009simulation}
that refinements and informative updates are equivalent.
What is most interesting about refinements is the ability to quantify over them.
This means that we can actually quantify over the set of informative updates, and to thus make
reasonings about all updates, or possible updates.\\
\\
van Ditmarsch and French also propose methods to augment existing logics with their quantification
in \cite{van2009simulation}.
van Ditmarsch, French and Pinchinat also combine their work to construct a modal logic augmented
with this quantification, the future event logic \cite{van2010future}.
These steps were the beginning to being able to quantify across informative updates and models.\\
\\
Hales unifies the work propose in \cite{van2009simulation}, by proposing an arbitrary action model
logic in \cite{hales13synthesis}.
Hales also completes the semantics for this new logic, by showing that quantification over
refinements is the same as quantifying over updates.\\
\\
These ideas are not directly related to update construction, but later they play a part in forming
models effectively.
We will employ the power of refinement quantification and the semantics given by Hales to
effectively construct action models \cite{hales13synthesis}.
