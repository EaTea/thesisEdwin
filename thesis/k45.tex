\section{Constructing event models in $\mathcal{K}45$ --- fixed by 7th October 2013}

We have discussed models in the axiom system of $\mathcal{K}$.
In particular, we have focused on approximating arbitrary models.
Our approximations can fulfill postconditions and the effects of event model execution, up to modal
depth $n$.
We will replicate a similar result of event model approximation in the axiomatic system
of $\mathcal{K}45$.
This chapter details our results for our event model approximation in the axiomatic system
$\mathcal{K}45$.

\subsection{Background --- fixed by 5 September 2013}

Previously, we constructed models that were tree-like in their nature.
We defined many concepts used to describe trees, such as a ``root", ``subtrees" and ``forests".
This allowed us to approximate arbitrary models in $\mathcal{K}$, to model any post-condition we
were interested in.

\begin{figure}[ht!]
\centering
\begin{tikzpicture}[-,>=stealth',shorten >=1pt,auto,node distance=2cm,
			thick]

		\node[vertex] (1) {$\evs_0$};
		\node[vertex] (2) [below left of=1] {$\evs_1$};
		\node[vertex] (3) [right of=2] {$\evs_2$};
		\path[edge]
					(1) edge node {} (2)
					(1) edge node {} (3);
\end{tikzpicture}
\caption{A very simple model that we our previous methodology constructed.
We can immediately see the concept of a ``root" node and ``leaf" nodes.}
\label{exampleModel}
\end{figure}

However, the models we previously generated are not so sensible under an epistemic interpretation.
Indeed, although we can approximate an epistemic event model and the informative update that it
encodes, our resultant model is not so easily interpreted within the original logical systems we
used.
At present, we can thus approximate any event model, whether it is interpretable under epistemic
logics or not.\\
\\
In order to discuss epistemic logic, we need to first impose more restrictions on what axioms our
models must now abide by.
Within a modal logic context, knowledge has a specific structure which results from the axioms
specified in the axiomatic system.
We will move closer to the mainstream axiomatic systems used in epistemic logic by moving to the
axiomatic system \AXKFF.
We define the axiomatic system of \AXKFF as follows

\begin{lemma} \label{axiomK45}
The axiomatisation \AXKFF contains the axioms and rules from \AXK (Definition
    \ref{axiomK}), as well as the following axioms
\begin{alignat*}{2}
  & \axFo && \quad \Box \phi \implies \Box \Box \phi \\
  & \axFi && \quad \Diamond \phi \implies \Box \Diamond \phi
\end{alignat*}
\end{lemma}

\begin{lemma} \label{axiomK45SoundComplete}
The axiomatisation \AXKFF is sound and complete with respect to the logic
$\lang$.
\end{lemma}

\AXKFF was first constructed by \FIXME and proved to be a sound and complete axiomatic
system by Grossi in \cite{grossi2007designing}.\\
\\
The impact these axioms have on our models is not immediately obvious.
We observe the following effects of the axioms
\begin{itemize}
	\item \axFo causes every relation to be transitive
	\item \axFi has more subtle consequences.
    Suppose $A$ is related to $B$.
    Then if $C$ is related to $A$, $C$ must also be related to $B$.
    A further consequence is that $B$ must also be related to $C$.
\end{itemize}

We demonstrate this through by comparing models in \AXK and \AXKFF and how we need to change
our models to obey these axioms.\\
\\
\FIXME comparative examples\\
\\
We can approximate models in \AXKFF using our previous method, but as we mentioned
the generated model is far from resembling an epistemic event model.
Models in the axiomatic system \AXKFF resemble epistemic event models much more closely.
Furthermore, any epistemic event model will obey the axioms of \AXKFF.\\
\\
An open question, related to our previous chapter --- can we now generate \AXKFF models to
approximate other \AXKFF models?

\subsection{Technical Preliminaries --- fixed by 11 September 2013}

We first define what a $B$-insane set is.
It is a necessary condition in order to generate our approximating \AXKFF models.

\begin{defn} \label{binsane}
	Let $\evM_\evT = ((\evS, \evR, \evpre), \evT)$ be a multi-pointed event model and $B \subseteq A$.
	We say that $\evM_\evT$ is $B$-insane if and only if for all $\evs \in \evT$,
  there is no $\evt \in \evS$ and $b \in B$ such that $\evs \evR_b \evt$.
\end{defn}

If $B = \{b\}$, we say that the set $\evM_\evT$ is $b$-insane.
Similarly if $B = A$ we say that $\evM_\evT$ is simply insane, which fulfills
our previous definition of insanity models (Definition \ref{insanity}).\\
\\
What does it mean for a set of states to be $B$-insane?
If $\evM_\evT$ is $B$-insane, then any agent $b$ in $B$ can distinguish between
action points, but only knows the preconditions of each action point.
They do not actually have any knowledge about what they know or any
introspection at each action point.\\
\\
We now define the main operation to construct event models in \AXKFF for the
purposes of realising a post condition.

\begin{defn} \label{makeEquivalence}
	Let $B \subseteq A$ and $\evM_\evT$ be a $B$-insane forest event model.
  For each $\evt \in \evT$ let $\evM_\evt$ be a tree event model in \AXKFF.
  Then the operation $I_B(\evM_\evT)$ constructs the event model $\evM'_{\evT'} =
  ((\evS',\evR',\evpr'),\evT')$ such that
  \begin{itemize}
    \item $\evS' = \evS$
    \item $\evR'_a = \evR_a$ if $a \notin B$
    \item $\evR'_a = \evR_a \cup \{(\evs, \evt) | \evs, \evt \in \evT\}$ if $a
    \in B$
    \item $\evpr' = \evpr$
    \item $\evT' = \evT$
  \end{itemize}
\end{defn}

We will say that $I_B$ is an operation that makes all the action
points in a ``distinguished" set of a $B$-insane event model ``equivalent" to a
group of agents $B$.
That is, any agent in $B$ considers each action point $\evs \in \evT$ to be the
same.

\FIXME example here\\
\\
The addition of our new operator results in us having the following language for
event model construction in \AXKFF.
\[
  \evM_\evT = \evM_\evT \text{ } | \text{ }\evM_\evT \to \evM'_{\evT'} \text{ }|
  \text{ } \evM_\evT \sqcup \evM'_{\evT'} \text{ } | \text{ } E_B(\evM_\evT)
\]
where $\evM_\evT, \evM'_\evT'$ are multi-pointed event models and $B \subseteq
A$.\\
\\
We will now show that this language always constructs models in \AXKFF, and
furthermore that some post condition in the language \lang can be realised by
some event model that our language can construct.

\subsection{Some Interesting Technical Results --- fixed by 27 September 2013}

\FIXME go through the process
