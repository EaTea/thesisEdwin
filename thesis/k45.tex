\section{Constructing event models in $\AXKFF$ --- fixed by 7th October 2013}

We have discussed models in the axiom system of $\AXK$.
In particular, we have focused on approximating arbitrary models.
Our approximations can fulfill postconditions and the effects of event model execution, up to modal
depth $n$.
We will replicate a similar result of event model approximation in the axiomatic system
of $\AXKFF$.
This chapter details our results for our event model approximation in the axiomatic system
$\AXKFF$.

\subsection{Background --- fixed by 5 September 2013}

Previously, we constructed models that were tree-like in their nature.
We defined many concepts used to describe trees, such as a ``root", ``subtrees" and ``forests".
This allowed us to approximate arbitrary models in $\AXK$, to model any post-condition we
were interested in.

\begin{figure}[ht!]
\centering
\begin{tikzpicture}[-,>=stealth',shorten >=1pt,auto,node distance=2cm,
			thick]

		\node[vertex] (1) {$\evs_0$};
		\node[vertex] (2) [below left of=1] {$\evs_1$};
		\node[vertex] (3) [right of=2] {$\evs_2$};
		\path[edge]
					(1) edge node {} (2)
					(1) edge node {} (3);
\end{tikzpicture}
\caption{A very simple model that we our previous methodology constructed.
We can immediately see the concept of a ``root" node and ``leaf" nodes.}
\label{exampleModel}
\end{figure}

However, the models we previously generated are not so sensible under an epistemic interpretation.
Indeed, although we can approximate an epistemic event model and the informative update that it
encodes, our resultant model is not so easily interpreted within the original logical systems we
used.
At present, we can thus approximate any event model, whether it is interpretable under epistemic
logics or not.\\
\\
In order to discuss epistemic logic, we need to first impose more restrictions on what axioms our
models must now abide by.
Within a modal logic context, knowledge has a specific structure which results from the axioms
specified in the axiomatic system.
We will move closer to the mainstream axiomatic systems used in epistemic logic by moving to the
axiomatic system $\AXKFF$.
We define the axiomatic system of $\AXKFF$ as follows

\begin{lemma} \label{axiomK45}
The axiomatisation $\AXKFF$ contains the axioms and rules from $\AXK$ (Definition
    \ref{axiomK}), as well as the following axioms
\begin{alignat*}{2}
  & \axFo && \quad \Box \phi \Rightarrow \Box \Box \phi \\
  & \axFi && \quad \Diamond \phi \Rightarrow \Box \Diamond \phi
\end{alignat*}
\end{lemma}

\begin{lemma} \label{axiomK45SoundComplete}
The axiomatisation $\AXKFF$ is sound and complete with respect to the logic
$\lang$.
\end{lemma}

$\AXKFF$ was first constructed by \FIXME and proved to be a sound and complete axiomatic
system by Grossi in \cite{grossi2007designing}.\\
\\
The impact these axioms have on our models is not immediately obvious.
We observe the following effects of the axioms
\begin{itemize}
	\item $\axFo$ causes every relation to be transitive
	\item $\axFi$ has more subtle consequences.
    Suppose $A$ is related to $B$.
    Then if $C$ is related to $A$, $C$ must also be related to $B$.
    A further consequence is that $B$ must also be related to $C$.
\end{itemize}

We demonstrate this in Figure \ref{k45VsKModels} by comparing models in $\AXK$ and $\AXKFF$ and how we need to change
our models to obey these axioms.

\begin{figure}[ht!]
\centering
\begin{subfigure}[b]{.45\textwidth}
\centering
\begin{tikzpicture}[->,>=stealth',shorten >=1pt,auto,node distance=2cm,
      thick]

    \node[vertex] (1) {$\alpha$};
    \node[vertex] (2) [right of=1] {$\beta$};
    \node[vertex] (3) [above right of=2] {$\gamma$};
    \node[vertex] (4) [below right of=2] {$\delta$};
    \path[edge]
          (1) edge node {} (2)
          (2) edge node {} (3)
          (2) edge node {} (4);
\end{tikzpicture}
\caption{Suppose for our Kripke model $\krMo$ (which obeys the rules and axioms in $\AXK$)
  that $\krMo_\alpha \models p$, $\krMo_\beta \models q$, $\krMo_\gamma \models r$ and
  $\krMo_\delta \models q \land p$.
We can say that $\krMo_\alpha \models \Box q$, but we cannot say that $\krMo_\alpha
\models \Box \Box q$.
Then $\axFo$ is disobeyed and thus $\krMo$ is not a $\AXKFF$ event model.}
\label{kmodel}
\end{subfigure}
~
\begin{subfigure}[b]{.45\textwidth}
\centering
\begin{tikzpicture}[->,>=stealth',shorten >=1pt,auto,node distance=2cm,
      thick]

    \node[vertex] (1) {$\alpha$};
    \node[vertex] (2) [right of=1] {$\beta$};
    \node[vertex] (3) [above right of=2] {$\gamma$};
    \node[vertex] (4) [below right of=2] {$\delta$};
    \path[edge]
          (1) edge node {} (2)
              edge node {} (3)
              edge node {} (4)
          (2) edge node {} (3)
              edge node {} (4)
              edge [loop right] node {} (2)
          (3) edge node {} (2)
              edge node {} (4)
              edge [loop above] node {} (3)
          (4) edge node {} (2)
              edge node {} (3)
              edge [loop below] node {} (4);
\end{tikzpicture}
\caption{Now our Kripke model $\krMo'$ obeys $\AXKFF$.
Let $\krMo'_\alpha \models p, \krMo'_\beta \models q, \krMo'_\gamma \models q \land r$ and
  $\krMo'_\delta) \models q \land p$.
Here, we see that (by $\axFo$) in order to say $\krMo'_\alpha \models \Box q$, we must also be
able to say $\krMo'_\alpha \models \Box \Box q$.}
\label{k45model}
\end{subfigure}
\caption{We compare two Kripke models, $\krMo$ (Subfigure \ref{kmodel}) and
  $\krMo'$ (Subfigure \ref{k45model}).}
\label{k45VsKModels}
\end{figure}

In Figure \ref{k45VsKModels} we compare a Kripke model in $\AXK$ to a model in $\AXKFF$.
We have already briefly discussed how $\axFo$ affects the reasonings we can make on the $\krMo_\alpha$ versus
$\krMo'_\alpha$.
Another observation we can make is with regards to the effect of $\axFi$ on $\krMo'_\alpha$ (Subfigure
\ref{k45model}).
By $\axFi$ for any formula $k \in \lang$ such that $\krMo'_\alpha \models
\Diamond k$, we have that $\krMo'_\alpha \models \Box \Diamond k$.
This causes points $\beta, \gamma$ and $\delta$ to become a completely connected component,
with an edge running to each one of them, as well as reflexive edges for each point.\\
\\
Up to now, what we have shown is an already well-defined idea for whether a Kripke model obeys the axioms of $\AXKFF$.
Unfortunately, this does not translate well to event models due to the differences in their
semantics.
In order to determine if an event model obeys the axioms of $\AXKFF$, we will formalise the ``model"
implications we previously discussed into conditions on the structure of the model (that is, the
frame of the model --- see Definition \ref{frame}).
These are the frame conditions of $\AXKFF$, and we will show that an event model that fulfills these
frame conditions, when executed on a Kripke model that obeys the axioms of $\AXKFF$ will preserve
these axioms post execution.

\begin{lemma} \label{lemma:k45frameconditions}
	Let $F = (S,R)$ be a frame.
	We say that $F$ is a $\AXKFF$ frame if it fulfills the following conditions:
	\begin{itemize}
		\item if $R$ is a transitive relation, then $\axFo$ is valid in any model where $F$ is its
			underlying structure --- that is, for $s, t, u \in S$ then $s R t \land t R u \implies s R u$
		\item if $R$ is a Euclidean relation, then $\axFi$ is valid in any model where $F$ is its
			underlying structure --- that is, for $s, t, u \in S$ then $s R t \land s R u \implies t R u$
	\end{itemize}
\end{lemma}

This is given as an exercise in \cite{fagin1995reasoning}, and we leave it as an exercise to the
reader to show that this holds.
We can use the concept of an $\AXKFF$ frame to define a $\AXKFF$ event model.

\begin{defn} \label{defn:k45eventModel}
	Suppose $\evM = (\evS, \evR, \evpr)$ is an event model.
	We say that $\evM$ is a {\em $\AXKFF$ event model} if $F = (\evS, \evR)$ is a $\AXKFF$ frame.
\end{defn}

The usefulness of a {\em $\AXKFF$ event model} is twofold.
Firstly, it is a structure which has some correspondence to the axioms of $\AXKFF$.
This means it has some epistemic meaning and can be interpreted in a setting that resembles
reasonings about knowledge and belief.
Secondly, we can show that if $\evM_\evT$ is a $\AXKFF$ event model then it preserves the axioms of
$\AXKFF$.

\begin{lemma} \label{lemma:k45preserved}
	Let $\krMo_T$ is a $\AXKFF$ Kripke model and $\evM_\evT$ is a $\AXKFF$ event model, then
	$\krMo_T \otimes \evM_\evT$ is a $\AXKFF$ Kripke model.
\end{lemma}
\begin{proof}
	\FIXME go proof gogogogogogo
\end{proof}

This shows us that what we consider a $\AXKFF$ event model behaves as we expect --- if we execute
its update on a $\AXKFF$ Kripke model, the resulting Kripke model is also a $\AXKFF$ Kripke model.\\
\\
We can approximate informative updates using our previous method (detailed in Chapter
\ref{chapter:Multiagent}, but these updates do not preserve
axioms $\axFo$ and $\axFi$.
The resulting model barely resembles an epistemic model.
The approach in this chapter constructs event models that, after their execution on a model that
fulfills the conditions of $\AXKFF$, will preserve those conditions.\\
\\
An open question, related to our previous chapter --- can we now generate $\AXKFF$ models to
approximate other $\AXKFF$ models?
\FIXME this paragraph is unclear.

\subsection{Technical Preliminaries --- fixed by 11 September 2013}

We first define what a $B$-insane set is.
It is a necessary condition in order to generate our approximating $\AXKFF$ models.

\begin{defn} \label{binsane}
	Let $\evM_\evT = ((\evS, \evR, \evpre), \evT)$ be a multi-pointed event model and $B \subseteq A$.
	We say that $\evM_\evT$ is {\em $B$-insane} if and only if for all $\evs \in \evT$,
	for all $b \in B$ we have that $\evs \evR_b = \evR_b \evs = \varnothing$.
\end{defn}

If $B = \{b\}$, we say that the set $\evM_\evT$ is $b$-insane.
Similarly if $B = A$ we say that $\evM_\evT$ is simply insane, which fulfills
our previous definition of insanity models (Definition \ref{insanity}).\\
\\
What does it mean for a set of states to be $B$-insane?
If $\evM_\evT$ is $B$-insane, then any agent $b$ in $B$ can distinguish between
action points, but only knows the preconditions of each action point.
They do not actually have any knowledge about what they know or any
introspection at each action point.
This is demonstrated in Figure \ref{bInsaneExample}.
\FIXME might need a pagebreak here.

\begin{figure}[ht!]
\centering
\begin{tikzpicture}[->,>=stealth',shorten >=1pt,auto,node distance=2cm,
      thick]

    \node[vertex] (1) {$\alpha$};
    \node[vertex] (2) [below of=1] {$\beta$};
    \node[vertex] (3) [right of=1] {$\gamma$};
    \node[vertex] (4) [below of=3] {$\delta$};
    \path[edge]
          (1) edge node {$B$} (2)
          (3) edge node {$C$} (4)
          (4) edge [loop below] node {$B$} (4);
\end{tikzpicture}
\caption{Consider our event model $\evM$. Let $B$ and $C$ be mutually exclusive
  subsets of $A$. Here, $\evM_\gamma$
  is $B$-insane, since there are no outgoing or incoming $B$-edges from these event points.
This is easily contrasted with $\evM_\alpha$, which has an outgoing $B$-edge, $\evM_\beta$, which
has an incoming $B$-edge, and $\evM_\delta$ which has a reflexive $B$-edge.}
\label{bInsaneExample}
\end{figure}

We now define the main operation to construct event models in $\AXKFF$ for the
purposes of realising a post condition.

\begin{defn} \label{makeEquivalence}
	Let $B \subseteq A$ and $\evM_\evT$ be a multi-pointed $B$-insane event model.
  For each $\evt \in \evT$ let $\evM_\evt$ be a tree event model in $\AXKFF$.
  Then the operation $I_B(\evM_\evT)$ constructs the event model $\evM'_{\evT'} =
  ((\evS',\evR',\evpr'),\evT')$ such that
  \begin{itemize}
    \item $\evS' = \evS$
    \item $\evR'_a = \evR_a$ if $a \notin B$
    \item $\evR'_a = \evR_a \cup \{(\evs, \evt) | \evs, \evt \in \evT\}$ if $a
    \in B$
    \item $\evpr' = \evpr$
    \item $\evT' = \evT$
  \end{itemize}
\end{defn}

We will say that $I_B$ is an operation that makes all the action
points in a ``distinguished" set of a multi-pointed event model ``equivalent" to a
group of agents $B$.
That is, any agent in $B$ considers each action point $\evs \in \evT$ to be the
same.

\begin{figure}[ht!]
\centering
\begin{subfigure}[b]{.45\textwidth}
\centering
\begin{tikzpicture}[->,>=stealth',shorten >=1pt,auto,node distance=2cm,
      thick]

    \node[vertex] (1) {$\alpha$};
    \node[vertex] (2) [below of=1] {$\beta$};
		\node[vertex] (5) [right of=2] {$\tau$};
    \node[vertex] (4) [right of=5] {$\delta$};
    \node[vertex] (3) [above of=4] {$\gamma$};
    \path[edge]
          (1) edge node {$B$} (2)
          (3) edge node {$C$} (4)
          (4) edge [loop below] node {$B$} (4);
\end{tikzpicture}
\caption{Consider the model $\evMo_\evT$, with $\evT = \{\tau, \gamma\}$.}
\label{beforeOperation}
\end{subfigure}
~
\begin{subfigure}[b]{.45\textwidth}
\centering
\begin{tikzpicture}[->,>=stealth',shorten >=1pt,auto,node distance=2cm,
      thick]

    \node[vertex] (1) {$\alpha$};
    \node[vertex] (2) [below of=1] {$\beta$};
		\node[vertex] (5) [right of=2] {$\tau$};
    \node[vertex] (4) [right of=5] {$\delta$};
    \node[vertex] (3) [above of=4] {$\gamma$};
    \path[edge]
          (1) edge node {$B$} (2)
					(2) edge [loop below] node {$B$} (2)
					(5) edge [loop below] node {$B$} (5)
							edge node {$B$} (3)
          (3) edge node {$C$} (4)
							edge [loop above] node {$B$} (3)
							edge node {} (5)
          (4) edge [loop below] node {$B$} (4);
\end{tikzpicture}
\caption{Consider our new model $I_B(\evMo_\evT)$, with $\evMo_\evT$ defined as in Subfigure
\ref{beforeOperation}.}
\label{afterOperation}
\end{subfigure}
\caption{Subfigures \ref{beforeOperation} and \ref{afterOperation} demonstrate the effect of
applying our operation $I_B$ onto a model.
Note the }
\label{k45VsKModels}
\end{figure}

In order for us to use our previous operations in $\AXKFF$ we need to force further constraints on
some of our operators.
It is clear that $\sqcup$ will preserve the frame conditions of $\AXKFF$, but $\to_B$ does not.
We will redefine $\to_B$ to ensure that within our language it cannot create event models that are
are not $\AXKFF$ event models.

\begin{defn} \label{defn:k45:considers}
	Suppose $B \subseteq A$ is a subset of agents.
	Let $\evM_\evT = ((\evS,\evR,\evpr),\evT)$ and $\evM'_{\evT'} = ((\evS',\evR',\evpr'),\evT')$ be $\AXKFF$ event models.
	Furthermore, $\evM_\evT$ is a $B$-insane event model, and $\evM'_{\evT'}$ is defined such that for
	all $b$ in $B$, for all $\evt \in \evT'$, $\evt \evR'_b subseteq \evT'$.
	$\evM_\evT \to_B \evM'_{\evT'}$ is defined as in Definition \ref{considers}.
\end{defn}

Note the constraints we place on $\evM_\evT$ and $\evM'_{\evT'}$ being $B$-insane models.
These are necessary in order to preserve the frame conditions in Lemma
\ref{lemma:k45frameconditions}.

\newcommand{\EM}{\ensuremath{\eventClass}}

The addition of our new operator results in us having the following language for
event model construction in $\AXKFF$.
\[
	\evM_\evT ::= \aMod{M_s} \text{ } | \text{ }\evM_\evT \to_B \evM'_{\evT'} \text{ }|
  \text{ } \evM_\evT \sqcup \evM'_{\evT'} \text{ } | \text{ } I_B(\evM_\evT)
\]
where $\aMod{M_s} \in \insaneClass, \evM_\evT, \evM'_{\evT'}$ are multi-pointed event models and $B \subseteq
A$ is a subset of agents.
We will refer to our language as $\EM(\to,\sqcup,I)$. \\
\\
We will now show that this language always constructs models that preserve $\AXKFF$, and that any
event model that preserves the axioms of $\AXKFF$ can be constructed by our language.

\subsection{Some Interesting Technical Results}

To show that we can construct event models that preserve $\AXKFF$, there are two propositions that
must hold
\begin{enumerate}
	\item we can use our language to construct event models that will preserve $\AXKFF$
	\item an event model that preserves $\AXKFF$ can be constructed (up to $n$-bisimilarity) by our
		language
\end{enumerate}

Let us show that the first proposition --- that $\EM(\to,\sqcup,I)$ will construct $\AXKFF$ event
models.

\begin{lemma}
	Let $\evM_\evT$ be a pointed event model constructed from the language $\EM(\to,\sqcup,I)$.
	Then $\evM_\evT$ is a $\AXKFF$ event model.
\end{lemma}
\begin{proof}
	\FIXME prooooooooooooof
\end{proof}

We have shown that any event models generated from $\EM(\to,\sqcup, I)$ are
$\AXKFF$ event models.
Thus, we can guarantee that if $\evM_\evT$ is a model generated by
$\EM(\to,\sqcup,I)$ and $\krMo_T$ is a $\AXKFF$ Kripke model, $\krMo_T
\otimes \evM_\evT$ is also a $\AXKFF$ Kripke model.
This shows that the constructions we use --- $\EM(\to, \sqcup, I)$ are
sound, and will preserve the axioms of $\AXKFF$ after their execution.\\
\\
Let us turn to the second goal we have and suppose that we have an event model, $\evM_\evT
= ((\evS,\evR,\evpr),\evT)$ that is a $\AXKFF$ multi-pointed event model.
Now, consider $n$ some positive integer.
Can we construct an event model $\evM'_{\evT'} = ((\evS',\evR',\evpr'),\evT')$
using $\EM(\to,\sqcup,I)$ that is $n$-bisimilar to $\evM_\evT$?\\
\\
In order to show this to be possible, we must first conceptualise several
transformations to our models.
Firstly, we will define the idea of an unwound model.

\begin{defn}\label{def:unwoundModel}
  Suppose $\evM_\evT = ((\evS,\evR,\evpr),\evT)$ is a $\AXKFF$ multi-pointed
  event model.
  There is an event model, $\evM'_{\evT'} = ((\evS',\evR',\evpr'),\evT')$ such that
  \begin{itemize}
		\item there is a many-to-one mapping $f$ from $\evS'$ to $\evS$ such that $\evpr(\evs) \iff
			\evpr'(\evs')$, and its inverse $f^{-1}$ maps
			$\evS$ to the powerset of $\evS'$
		\item for all $\evs' \in \evS'$, for all $\evt$ in $f(\evs') \evR_a$, there is some unique $\evt' \in
			f^{-1}(\evt)$ such that $\evs' \evR'_a \evt'$
    \item for every $\evs$ in $\evS'$, if $a \in A$ and $\evR'_a \evs$ is non-empty then
    $\evs \evR'_a$ is empty
    \item for all $\evs \in \evS'$, if $a \in A$ and $\evp, \evq \in \evs \evR'_a$ then $\evp
    \evR'_a \evq$
		\item for all $\evs, \evt, \evu \in \evS'$, for $a \in A$ and $\evs \evR_a \evt$ and $\evt
			\evR_a \evu$ then $\evs \evR_a \evu$
  \end{itemize}
  We say that $\evM'_{\evT'}$ is the {\em unwound model} of $\evM_\evT$.
\end{defn}

\begin{lemma} \label{lemma:unwoundModel:bisimilar}
  Suppose $\evM_\evT = ((\evS,\evR,\evpr),\evT)$ is a $\AXKFF$ multi-pointed
  event model.
	Let $\evM'_{\evT'} = ((\evS',\evR',\evpr'),\evT')$ be the unwound model of $\evM_\evT$.
	Then $\evM'_{\evT'}$ is a $\AXKFF$ event model and $\evM'_{\evT'} \sim \evM_\evT$.
\end{lemma}
\begin{proof}
	By the definition of an unwound model, we can see that the frame conditions in Lemma 
	\ref{lemma:k45frameconditions} will automatically be fulfilled.
	Similarly, the mapping $f$, as defined in Definition \ref{def:unwoundModel}, ensure that event
	model bisimilarity as defined in Definition \ref{bisimEvent} also hold.
\end{proof}

The unwound model is an infinite pseudo-forest event model.
This means that the model somewhat resembles our original forest event models,
but with interior relations between the action points, and without the
restriction of finiteness.
We show an example of this in Figure \ref{generatedTreeExample}

\begin{figure}[ht!]
\centering
\caption{\FIXME Left subfigure should be original, right subfigure should be the
unwound model.} \label{generatedTreeExample}
\end{figure}

The unwound model is bisimilar to the original $\AXKFF$ event model.
However, as it may be an infinite event model (in terms of the number of action
points), it is undesirable to try and construct this model.
Furthermore, most of the time, we might only be interested in the
post-conditions of a event model that are up to some modal depth.
If we instead consider a finite subtree of the unwound model event
model, we can show that this submodel is $n$-bisimilar to $\evM_\evT$.

\begin{defn} \label{unwoundNModel}
  Suppose $\evM_\evT = ((\evS,\evR,\evpr),\evT)$ is a $\AXKFF$ multi-pointed
  event model, and let $\evM'_{\evT'} = ((\evS', \evR', \evpr'),\evT')$ be the unwound model of $\evM_\evT$.
  Let $n$ be some positive integer.
	Suppose $\evM^n_{\evT^n} = ((\evS^n,\evR^n,\evpr^n),\evT^n)$ is the submodel of $\evM'_{\evT'}$ such that
  \begin{itemize}
    \item $\evS^n \subseteq \evS'$
    \item $\evR^n \subseteq \evR'$
    \item $\evpr^n \subseteq \evpr'$
    \item $\evT^n = \evT'$
    \item for every $\evs \in \evS$ there is some $\evt \in \evT$ such that a
    sequence of worlds of length $k \leq n$, $\evs_1, \evs_2, \ldots, \evs_k \in \evS$ and agents
    $a_1,a_2, \ldots, a_{k-1}$ where
    \[
      \evt = \evs_1 \evR_{a_1} \evs_2, \evs_2 \evR_{a_2} \evs_3, \ldots,
      \evs_{k-1} \evR_{a_{k-1}} \evs_k = \evs
    \]
  \end{itemize}
  We say that $\evM^n_{\evT^n}$ is the {\em unwound $n$-model} of $\evM_\evT$.
\end{defn}

\begin{lemma} \label{lemma:unwoundNModelNBisimilar}
  Suppose $\evM_\evT = ((\evS,\evR,\evpr),\evT)$ is a $\AXKFF$ multi-pointed
  event model.
  Let $n$ be some positive integer and $\evM^n_{\evT^n} = ((\evS^n,\evR^n,\evpr^n),\evT^n)$ be the
	unwound $n$-model of $\evM_\evT$.
	Then for every point $\evt \in \evT^n$, we have that $\evM^n_\evt \sim_n \evM'_{\evt}$; that is,
	that $\evM^n_{\evt}$ is $n$-bisimilar to $\evM'_{\evT'}$ and thus is
  $n$-bisimilar to $\evM_\evT$.
\end{lemma}
\begin{proof}
	We will induct over $n$ to show that this holds.
	We hypothesise that for every action point $\evt \in \evT^n$, $\evM^n_\evt \sim_k \evM'_{\evt}$
	for some $0 \leq k \leq n$.\\
	\\
	We begin with our base case of $k = 0$.
	Since $\evT' = \evT^n$ then $\evt \in \evT^n \Rightarrow \evt \in \evtT'$.
	It is clear that $\evt$ is $0$-bisimilar to itself, and therefore we conclude that $\evM^n_\evt
	\sim_0 \evM'_\evt$.\\
	\\
	Now, suppose that for $k = m$, our hypothesis holds.\\
	\\
	We will show that it now holds for $k = m+1$.
	We have already shown that $\evM^n_\evt \sim_{k-1} \evM'_\evt$, through the inductive step.
	We must now show both of {\bf $k$-forth-$a$} and {\bf $k$-back-$a$} as in Definition
	\ref{nBisimEvent}.\\
	\\
	Let us show {\bf $k$-forth-$a$}.
	Let $a \in A$, and suppose that $\evs \in \evt \evR^n_a$.
	We want to show there is some $\evs' \in \evt \evR'_a$ such that $\evs' \sim_{k-1} \evs$.
	To show that this is so, we would require that $\evs' \sim_{k-2} \evs$, and that {\bf
	$k-1$-forth-$a$} and {\bf $k-1$-back-$a$} both hold at $\evs$.\\
	\\
	We can continue this argument until we reach some new point $\evs \in \evS^n$ that we require to be
	$0$-bisimilar to some other point $\evs' \in \evS'$.
	It is clear that these new points are 0-bisimilar, that is $\evs \sim_0 \evs'$, which allows us to
	claim that {\bf $k$-forth-$a$} is originally satisfied at $\evt$.\\
	\\
	We can make a similar argument for {\bf $k$-back-$a$} and through this we can satisfy the
	conditions of Definition \ref{nBisimEvent}, allowing us to claim that for all $\evM^n_\evt \sim_k
	\evM'_{\evt}$.
\end{proof}

This submodel of the unwound model is $n$-bisimilar to our original $\evM_\evT$
and is finite, which are both desirable properties.
Using our previous example, we show in Figure \ref{genSubtreeExample} the
finite submodel that is $n$-bisimilar.

\begin{figure}[ht!]
\centering
\caption{\FIXME Show finite version of figure \ref{generatedTreeExample}} \label{genSubtreeExample}
\end{figure}

It remains to be shown that we can actually construct an unwound $n$-model using
$\EM(\to,\sqcup,I)$.
We will now show that we can actually construct any unwound $n$-model using $\EM(\to,
\sqcup, I)$.

\begin{lemma} \label{unwoundNModelGenerated}
  Suppose $\evM_\evT = ((\evS,\evR,\evpr),\evT)$ is a $\AXKFF$ multi-pointed
  event model, and let $\evM^n_{\evT^n} = ((\evS^n, \evR^n, \evpr^n),\evT^n)$ be the
  unwound $n$-model of $\evM_\evT$.
  $\evM^n_{\evT^n}$ can be constructed by $\EM(\to, \sqcup, I)$.
\end{lemma}
\begin{proof}
	A proof that this condition holds is amenable to a proof  by induction on the structure of the unwound $n$-model.
	The induction hypothesis is as follows --- for any point $\evs \in \evS^n$, it is possible to construct the submodel of
	the unwound $n$-model that is ``below" that point.
	That is, one can construct the submodel of the unwound $n$-model that contains $\evs$ and every
	point and relation that is outgoing from $\evs$, and every point and relation that is outgoing
	from those points, and so on until there are no more points reachable in $\evM^n_{\evs}$.\\
	\\
	Our base case is when $\evs$ has no outgoing relations.
	Then $\evM^n_{\evs} = ((\{\evs\}, \varnothing, \{(\evs,\evpr(\evs))\}),\{\evs\})$, which is an
	insanity model that is used as an atom in this event model construction method.\\
	\\
	Now, consider some arbirtrary point $\evs \in \evS^n$, and suppose that one can construct the submodels
	``below" each point $\evt \in \evs \evR^n_a$, for every $a \in A$, except for any $a$ where
	$\evR^n_a
	\evs$ is nonempty.
	Since in the original model $\evM^n$, $\evR^n_a \evt$ being nonempty implied that $\evt evR^n_a$
	is a nonempty set and we can thus say that the submodel $\evM^n_\evt$ is $a$-insane.
	Let $\evT^s = \evs \evR_a$, and consider the model
	\[
		\evM^s_{\evT^s} = \bigsqcup_{\evt \in \evT} \evM^n_\evt
	\]
	One can then use $I_a(\evM^s_{\evT^s})$ to construct what is essentially an equivalence relation
	over $\evT^s$.
	Since it has already been argued that $\evM^s_{\evT^s}$ is $a$-insane, the operation $I_a$ is
	valid to execute.\\
	\\
	Now, consider
	\[
		\aMod{M}_\evs = ((\{\evs\}, \varnothing, \{(\evs,\evpr(\evs))\}),\{\evs\})
	\]
	This model is also $a$-insane, since it has no outgoing $a$-edges.
	Furthermore, for event model $I_a(\evM^s_{\evT^s})$, for every $\evt \in \evT^s$, $\evt \evR^s_a =
	\evT^s$.
	This permits the use of the consider operation, as defined in Definition
	\ref{defn:k45:considers}, allowing the construction of the model
	\[
		\evM^{n'}_{\evT^{n'}} = \aMod{M}_\evs \to_a I_a(\evM^s_{\evT^s})
	\]
	\FIXME show that these models are the same?
\end{proof}

Using this result, we can now construct, up to $n$-bisimilarity, an
approximation of any update using semantically meaningful operations.
This means that we can approximate any update, with the size of our resulting
model dependent on the depth of the post-conditions we are interested in.
A corollary to this result follows with regards to the single agent case.

\begin{corr}
  Suppose $\evM_\evT = ((\evS,\evR,\evpr),\evT)$ is a $\AXKFF$ multi-pointed
  event model involving only one agent, $a$.
  We can construct $\evM'_{\evT'} = ((\evS',\evR',\evpr'),\evT')$ by
  $\EM(\to,\sqcup,I)$ where $\evM'_{\evT'} \sim \evM_\evT$.
\end{corr}
\begin{proof}
	\FIXME proof, it's not safe to go alone
\end{proof}
