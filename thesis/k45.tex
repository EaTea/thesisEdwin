\section{Constructing event models in $\mathcal{K}45$ --- fixed by 7th October 2013}

We have discussed models in the axiom system of $\mathcal{K}$.
In particular, we have focused on approximating arbitrary models.
Our approximations can fulfill postconditions and the effects of event model execution, up to modal
depth $n$.
We will replicate a similar result of event model approximation in the axiomatic system
of $\mathcal{K}45$.
This chapter details our results for our event model approximation in the axiomatic system
$\mathcal{K}45$.

\subsection{Background --- fixed by 5 September 2013}

Previously, we constructed models that were tree-like in their nature.
We defined many concepts used to describe trees, such as a ``root", ``subtrees" and ``forests".
This allowed us to approximate arbitrary models in $\mathcal{K}$, to model any post-condition we
were interested in.

\begin{figure}[ht!]
\centering
\begin{tikzpicture}[-,>=stealth',shorten >=1pt,auto,node distance=2cm,
			thick]

		\node[vertex] (1) {$\evs_0$};
		\node[vertex] (2) [below left of=1] {$\evs_1$};
		\node[vertex] (3) [right of=2] {$\evs_2$};
		\path[edge]
					(1) edge node {} (2)
					(1) edge node {} (3);
\end{tikzpicture}
\caption{A very simple model that we our previous methodology constructed.
We can immediately see the concept of a ``root" node and ``leaf" nodes.}
\label{exampleModel}
\end{figure}

However, the models we previously generated are not so sensible under an epistemic interpretation.
Indeed, although we can approximate an epistemic event model and the informative update that it
encodes, our resultant model is not so easily interpreted within the original logical systems we
used.
At present, we can thus approximate any event model, whether it is interpretable under epistemic
logics or not.\\
\\
In order to discuss epistemic logic, we need to first impose more restrictions on what axioms our
models must now abide by.
Within a modal logic context, knowledge has a specific structure which results from the axioms
specified in the axiomatic system.
We will move closer to the mainstream axiomatic systems used in epistemic logic by moving to the
axiomatic system \AXKFF.
We define the axiomatic system of \AXKFF as follows

\begin{lemma} \label{k45axioms}
\FIXME the axioms of K45
\end{lemma}

\begin{lemma} \label{k45SoundComplete}
\FIXME k45 is sound and complete
\end{lemma}

$\mathcal{K}45$ was first constructed by \FIXME and proved to be a sound and complete axiomatic
system by \FIXME.\\
\\
The impact these axioms have on our models is not immediately obvious.
We observe the following effects of the axioms
\begin{itemize}
	\item \FIXME formatting --- 4 causes every relation to be transitive
	\item \FIXME formatting --- 5 causes every relation to need to relate to any other accessible
world
\end{itemize}

We demonstrate this through by comparing models in \AXK and \AXKFF and how we need to change
our models to obey these axioms.\\
\\
\FIXME comparative examples\\
\\
We can approximate models in \AXKFF using our previous method, but as we mentioned
the generated model is far from resembling an epistemic event model.
Models in the axiomatic system \AXKFF resemble epistemic event models much more closely.
Furthermore, any epistemic event model will obey the axioms of \AXKFF.\\
\\
An open question, related to our previous chapter --- can we now generate \AXKFF models to
approximate other \AXKFF models?

\subsection{Technical Preliminaries --- fixed by 11 September 2013}

\FIXME this should change to pointed event models

We first define what a $B$-insane set is.
It is a necessary condition in order to generate our approximating \AXKFF models.
Here, we say that $\evS$ is a finite set of action points from some event model.

\begin{defn} \label{binsane}
	Let $\evS$ be a finite set of action points and $B \subseteq A$.
	We say that $\evS$ is $B$-insane if and only if for all $\evs \in \evS$, there is no $\evt \in
\evS$ and $b
\in B$ such that $\evs \evR_b \evt$.
\end{defn}

If $B = \{b\}$, we say that the set $\evS$ is $b$-insane.
Similarly if $B = A$ we say that $\evS$ is simply insane.\\
\\
What does it mean for a set of states to be $B$-insane?
If $\evS$ is $B$-insane, then any agent $b$ in $B$ can distinguish between worlds, but only knows
facts at each world.
They do not actually have any knowledge about what they know at each world.\\
\\
\begin{defn} \label{makeEquivalence}
	Let $B \subseteq A$ and $\evS$ be a $B$-insane set of action points.
	We say that \FIXME needs finite event models but essentially put in the new operator for making an
equivalence class
\end{defn}

\FIXME example here\\
\\
\FIXME recap the language perhaps?

\subsection{Some Interesting Technical Results --- fixed by 27 September 2013}

\FIXME go through the process
