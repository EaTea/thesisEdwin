\chapter{Introduction} \label{chapter:intro}

Suppose Alice and Bob are two share traders who fervently watch the stock market. They
are particularly concerned about the fate of company ABC and are waiting for news on whether ABC is
doing well or poorly. In the middle of their stock market vigil, Alice is given a letter that says
ABC has done particularly well. Bob sees her open the letter and read its contents, but he does not
know what the letter is about.\\
\\
If we wish to formally model Alice and Bob's stock situation, we must consider how we can model
\begin{enumerate}
	\item the facts Alice and Bob know or believe before Alice obtained the letter
	\item the change in belief or knowledge described by Alice's letter and Bob's observations
	\item the updated facts Alice and Bob know or believe after the letter has been opened
\end{enumerate}

Epistemic modal logic is particularly concerned with modelling item one --- the study of knowledge,
belief and the interplays of uncertainty.
The facts and beliefs that Alice and Bob have within their possession prior to an update are well-studied phenomena.
For example, epistemic models of the situation prior to Alice seeing her letter are that both Alice
and Bob consider it feasible that company ABC is doing poorly, and they also consider it feasible
that company ABC is doing well.
However they cannot tell the difference between either of these possibilities.\\
\\
Modelling epistemic situations, in accordance with axiomatic schemes that
formally model knowledge, is a well-studied field.
The work in this area has been used in financial trading and games to reason about the knowledge
that agents in these systems have, as well as model checking and automated theorem proving.
Epistemic modal logic thus finds its uses when logical deductions about knowledge and belief must be
made, as well as proving that an agent knows or believes exactly some set of propositions at a
particular time step.\\
\\
This work is particularly concerned with items two and  three --- the update that Alice receives and Bob
indirectly infers from Alice's reading, as well as the new knowledge and beliefs that Alice and Bob
possess after the update.
Work in this area is relatively new, and dynamic epistemic modal logic is of particular concern
within this paper.
We are concerned with modelling changes and the effects of those changes.
In particular, being able to reason about the effects of changes, and to prove their properties and
effects is desirable.\\
\\
Updates can be represented using a relatively new concept --- action models, as
introduced by Baltag, Moss and Solecki \cite{baltag1998lpa}.
Most of the work has been centred on specifying action models and reasoning
about their properties.
However, very little has been done in constructing action models.
The difference between specification and construction is subtle, but has great
implications.
It is the difference between saying ``an update exists that reaches an epistemic
state" and ``this is how we construct the update that fulfils our epistemic state".\\
\\
Being able to construct action models offers us the ability to construct provable 
updates in automated systems.
Model checking and game theory systems would be much improved by formal methods
to construct updates that are provably correct.
These additional features for automated systems are useful in maintaining
a provable level of correctness and consistency of systems that deal with facts, knowledge and
beliefs.\\
\\
This work will document the different languages and the operations used in them to construct
different epistemic updates.
We will show a complementary work to Hales' work that constructs updates in the weakest modal logic,
$\AXK$, which is far removed from epistemic modal logic.
We then change axiom systems to a modal logic that reflects typical properties of knowledge,
$\AXKFF$, and show how we can construct updates in that logic.\\
\\
The rest of this paper is arranged as follows
\begin{itemize}
	\item Chapter \ref{lit_survey} gives an overview of the state of the art of epistemic modal logic
		and clearly outlines what problems the work in this thesis aims to solve
	\item Chapter \ref{chapter:prelim} gives the technical definitions, lemmas and theorems
		established in previous works
	\item Chapter \ref{chapter:Multiagent} shows results that compliment Hales' work in
		\cite{hales13synthesis}, particularly in constructing action models for epistemic goal
		achievement in $\AXK$
	\item Chapter \ref{chapter:k45} shows our results in constructing action models for epistemic goal
		achievement in $\AXKFF$, which is an axiom system that more closely resembles epistemic modal
		logic
\end{itemize}
