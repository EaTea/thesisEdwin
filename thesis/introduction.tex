\chapter{Introduction} \label{chapter:intro}

Suppose Angeline and Ben are two share traders who fervently watch the stock market. They
are particularly concerned about the fate of company ABC and are waiting for news on whether ABC is
doing well or poorly. In the middle of their stock market vigil, Angeline is given a letter that says
ABC has done particularly well. Ben sees her open the letter and read its contents, but he does not
know what the letter is about.\\
\\
Our interest in Angeline and Ben's predicament centers mainly around their knowledge of company ABC.
We might ask questions, such as
\begin{itemize}
	\item What can we say about Angeline and Ben's knowledge before the letter arrives?
	\item What could we say about their knowledge after the letter's arrival?
	\item What about Ben in particular --- what does he know, and what does he not know or is unsure about?
\end{itemize}
An epistemic model of this situation concerns itself with the information regarding company ABC, and
what each of our agents knows or believes about the situation.\\
\\
If we wish to formally model Angeline and Ben's stock situation, we must consider how we can model
\begin{enumerate}
	\item The facts Angeline and Ben know or believe before Angeline obtained the letter
	\item The change in belief or knowledge described by Angeline's letter and Ben's observations; and
	\item The updated facts Angeline and Ben know or believe after the letter has been opened
\end{enumerate}

Epistemic modal logic is particularly concerned with modelling item one --- the study of knowledge,
belief and the interplays of uncertainty.
For example, an epistemic model can model the possibilities that Angeline and Ben believe --- a
possibility where the company ABC has done well, and a possibility where the company ABC is
performing poorly.\\
\\
The epistemic models of the situation prior to Angeline seeing her letter further model how both Angeline
and Ben do not know whether company ABC is doing poorly, and that they also do not know whether
that company ABC is doing well.
The facts and beliefs that Angeline and Ben have within their possession prior to an update are well-studied phenomena.
However they cannot tell the difference between either of these possibilities.\\
\\
Modelling epistemic situations, in accordance with axiomatic schemes that
formally model knowledge, is a well-studied field.
The work in this area has been used in financial trading and games to reason about the knowledge
that agents in these systems have, as well as model checking and automated theorem proving.
Automated reasoning about knowledge, and maintaining it in a consistent state are two important 
Epistemic modal logic thus finds its uses when logical deductions about knowledge and belief must be
made, as well as proving that an agent knows or believes exactly some set of propositions at a
particular time step.\\
\\
This work is particularly concerned with how we model the letter that Angeline receives and Ben
indirectly infers from Angeline's reading of it.
Furthermore, we are interested in the new knowledge and beliefs that Angeline and Ben possess after the update.
What does Ben know after Angeline has read the letter?
How can we model the information in the letter so that Angeline is informed of company ABC's success,
and Ben is not?\\
\\
These questions that are related to items two and three --- changes in knowledge, and how we can
model the results of changes.
Work in this area is relatively new, and dynamic epistemic modal logic is of particular concern
within this paper.
In particular, being able to reason about the effects of changes, and to prove their properties and
effects is desirable.\\
\\
One method of representing updates is the concept of action models, as
introduced by Baltag, Moss and Solecki \cite{baltag1998lpa}.
A large body of work has been centred on specifying action models and reasoning
about their properties.
Our work looks at a different aspect of action models --- how to compose smaller action models
into larger ones under the constraints of a meaningful framework.\\
\\
In the case of Angeline and Ben, we might consider composition as a taking a series of sentences
regarding the performance of company ABC, such as ``Angeline knows that ABC is doing well", ``Angeline
believes that Ben believes ABC is doing poorly" or simply ``ABC is doing poorly".
Those sentences are composed together into a meaningful and useful letter that Angeline and Ben can
understand regarding the performance of company ABC (e.g. ``Angeline knows that ABC is doing well
and Angeline believes that Ben believes ABC is doing poorly").
These letters might in turn be sent to Angeline and Ben in bulk, as one large update to their knowledge
about ABC.\\
\\
Composition unifies some approaches to representing changes in knowledge with action models, namely
around representing action models as formulae and providing a language for updating action models.
Updates (the letters in the ABC example) can be written in terms of atomic updates (the sentences)
and operations which compose them together.
We specify a meaningful language that constructs larger action models out of atomic updates, and
show that it can effectively construct all updates.\\
\\
This work will document the different languages and the operations used in them to construct
different epistemic updates.
We will show a complementary work to Hales' \cite{hales13synthesis} work that constructs updates in the weakest modal logic,
$\AXK$, which is far removed from epistemic modal logic.
We then change axiom systems to a modal logic that reflects typical properties of knowledge,
$\AXKFF$, and show how we can construct updates in that logic.\\
\\
The rest of this paper is arranged as follows
\begin{itemize}
	\item Chapter \ref{lit_survey} gives an overview of the state of the art of epistemic modal logic
		and clearly outlines what problems the work in this thesis aims to solve
	\item Chapter \ref{chapter:prelim} gives the technical definitions, lemmas and theorems
		established in previous works
	\item Chapter \ref{chapter:Multiagent} shows results that compliment Hales' work in
		\cite{hales13synthesis}, particularly in constructing action models for epistemic goal
		achievement in $\AXK$
	\item Chapter \ref{chapter:k45} shows our results in constructing action models for epistemic goal
		achievement in $\AXKFF$, which is an axiom system that more closely resembles epistemic modal
		logic
\end{itemize}
