%-----------------------------------------------------------------------------%
%Packages%
\documentclass{cshonours}
\usepackage{amsmath, amsfonts, listings, amssymb, mathtools, amsthm} %Mathematical Expressions package
\usepackage{mathtools}
\usepackage{geometry}
\usepackage{float}
\usepackage{verbatim} %for code
\usepackage[pdftex]{graphics}
\usepackage{hyperref}
\usepackage{cleveref}
\usepackage{tikz}
\usepackage{comment}
\usepackage[nottoc]{tocbibind}
\usepackage{caption}
\usepackage{subcaption}
\usepackage{algorithm}
\usepackage{algpseudocode}

\usetikzlibrary{arrows,shapes}

%Graphis Extensions
\DeclareGraphicsExtensions{.png, .jpg}
\parindent 0pt

% Predefined things such as commands, etc.

\newtheorem{defn}{Definition}[chapter]
\newtheorem{thm}{Theorem}[chapter]
\newtheorem{lemma}{Lemma}[chapter]
\newtheorem{corr}{Corollary}[chapter]
\newtheorem{propn}{Proposition}[chapter]
\newtheorem*{remrk}{Remark}
\newcommand{\note}[1]{\textsc{\textbf{#1}}}

\numberwithin{equation}{chapter}

\newcommand{\cover}{\bigtriangledown}
\newcommand{\sqex}[1]{[{#1}]}
\newcommand{\anex}[1]{\langle {#1} \rangle}
\newcommand{\lang}{\ensuremath{L}}
\newcommand{\langRefine}{\ensuremath{\lang_{\forall}}}
\newcommand{\langActEx}{\ensuremath{\lang_{\otimes}}}
\newcommand{\langArbAct}{\ensuremath{\lang_{\otimes\forall}}}
\newcommand{\langProp}{\ensuremath{\lang_0}}

\newcommand{\langKFF}{\ensuremath{K45}}
\newcommand{\lKFFRefine}{\ensuremath{\langKFF_{\forall}}}
\newcommand{\lKFFActEx}{\ensuremath{\langKFF_{\otimes}}}
\newcommand{\lKFFArbAct}{\ensuremath{\langKFF_{\otimes\forall}}}
\newcommand{\lKFFProp}{\ensuremath{\langKFF_0}}

\newcommand{\AXK}{\ensuremath{\bf K}}
\newcommand{\AXKFF}{\ensuremath{\bf K45}}
\newcommand{\AXAML}{\ensuremath{\bf AML_{\AXK}}}
\newcommand{\AXRML}{\ensuremath{\bf RML_{\AXK}}}
\newcommand{\AXAAML}{\ensuremath{\bf AAML_{\AXK}}}

\newcommand{\axP}{\ensuremath{\bf P}}
\newcommand{\axK}{\ensuremath{\bf K}}
\newcommand{\axFo}{\ensuremath{\bf 4}}
\newcommand{\axFi}{\ensuremath{\bf 5}}
\newcommand{\axMP}{\ensuremath{\bf MP}}
\newcommand{\axNecK}{\ensuremath{\bf NecK}}
\newcommand{\axAN}{\ensuremath{\bf AN}}
\newcommand{\axAP}{\ensuremath{\bf AP}}
\newcommand{\axAC}{\ensuremath{\bf AC}}
\newcommand{\axAK}{\ensuremath{\bf AK}}
\newcommand{\axAU}{\ensuremath{\bf AU}}
\newcommand{\axNecA}{\ensuremath{\bf NecA}}
\newcommand{\axR}{\ensuremath{\bf R}}
\newcommand{\axRP}{\ensuremath{\bf RP}}
\newcommand{\axRK}{\ensuremath{\bf RK}}
\newcommand{\axRComm}{\ensuremath{\bf RComm}}
\newcommand{\axRDist}{\ensuremath{\bf RDist}}
\newcommand{\axNecR}{\ensuremath{\bf NecR}}

\newcommand{\krMo}{M}
\newcommand{\aMod}[1]{
  \ensuremath {\mathrm{\mathsf{#1}}}
}
\newcommand{\evMo}{\aMod{N}}
\newcommand{\evM}{\evMo}
\newcommand{\evMM}{\aMod{M}}
\newcommand{\evS}{\aMod{S}}
\newcommand{\evR}{\aMod{R}}
\newcommand{\evpr}{\aMod{pre}}
\newcommand{\evpre}{\evpr}
\newcommand{\evT}{\aMod{T}}
\newcommand{\evU}{\aMod{U}}
\newcommand{\evr}{\aMod{r}}
\newcommand{\evs}{\aMod{s}}
\newcommand{\evt}{\aMod{t}}
\newcommand{\evu}{\aMod{u}}
\newcommand{\evv}{\aMod{v}}
\newcommand{\evp}{\aMod{p}}
\newcommand{\evq}{\aMod{q}}

\newcommand{\kripkeClass}{\mathcal{K}}
\newcommand{\eventClass}{\mathcal{AM}}
\newcommand{\EM}{\ensuremath{\eventClass}}
\newcommand{\insaneClass}{\eventClass_{IN}}
\newcommand{\publicAnnClass}{\eventClass_{PA}}
\newcommand{\treeClass}{\eventClass_{TR}}
\newcommand{\forestClass}{\eventClass_{FOR}}

\newcommand{\mDep}{ {\bf MD}}

\newcommand{\FIXME}{{\bf FIXME}}
% Drawings of frames

\tikzstyle{vertex}=[circle,draw=black!100,thick,fill=black!25,minimum size=20pt,inner sep=0pt]
\tikzstyle{selected vertex} = [vertex, fill=red!24]
\tikzstyle{edge} = [draw,thick,->]
\tikzstyle{weight} = [font=\small]

\captionsetup{width=0.8\textwidth}

%-----------------------------------------------------------------------------%
%Document%

\begin{document}

Note: Assume $A$ is a finite set of agents and $L$ is a logical language for the purposes of these
notes.\\
\\
We want to develop a language that can compose all action models, up to $n$-bisimilarity.
The aim is to maintain the axioms $\axK, \axD, \axFo$ and $\axFi$.\\
\\
Current problems with my approach centre around difficulty in showing $n$-bisimilarity, particularly
because of adding in new relations onto existing states.
Example: suppose $\evM_\evs = ((\evS, \evR, \evpr), \evs)$ and $\evM'_{\evs'} = ((\evS', \evR',
\evpr'), \evs')$ are action models.
Then $\evM_\evs \to \evM'_{\evs'}$ adds relations into $\evR_a$ for every $a \in A$.
This can become increasingly difficult to show $n$-bisimilarity in a complex structure such as those
defined in my thesis.\\
\\
Another problem involves the atoms in this language.
Currently, they do not fulfill $\axD$.
We need a new atomic action model --- a prime candidate is the public announcement.
However, it is difficult to achieve the right level of granularity in KD45.
(Sometimes we do not want reflexive edges on a node at all, other times we want some agens to have
reflexive edges).\\
\\
Addressing the atoms issue is best done by switching to secret announcements.
We define secret announcements as follows.
\begin{defn}
	Let $B \subseteq A$ be some finite set of agents and $\phi \in L$ be some sentence in $L$.
	Let $\evM_\evT = ((\evS, \evR, \evpr), \evT)$ be a multi-pointed action model such that
	\begin{itemize}
		\item $\evS = \{\evs, \evt\}$
		\item $\evR_a = \{(\evs, \evs), (\evt, \evt)\}$ if $a \in B$
		\item $\evR_a = \{(\evs, \evt), (\evt, \evt)\}$ if $a \notin B$
		\item $\evpr(\evs) = \phi$ and $\evpr(\evt) = \top$
		\item $\evT = \{\evs\}$
	\end{itemize}
	We say $\evM_\evT$ is the {\em secret announcement of \phi to $B$}.
	We denote this as $\phi!_B$.
\end{defn}

These secret announcements are more flexible than public announcements (the set of secret
announcements does not contain the set of public announcements, since $\phi!_A$ still contains two
action points and will result in the simultaneous (?) announcement of $\top$ and $\phi$ to all
agents).\\
\\
To address the first issue, try changing operations?
Maintain the $I_B$ operator used in my thesis.
Add a new operator $\mathcal{S}_B$ as follows.

\begin{defn}
  Let $\evM_\evT = ((\evS, \evR, \evpr), \evT)$ be a multipointed action model,
  $B \subseteq A$ and $\phi \in L$.
  Then $\evM'_{\evs'} = \mathcal{S}(\evM_\evT, \phi) = ((\evS', \evR', \evpr'),
      {\evs'})$ is an action model such that
  \begin{itemize}
    \item $\evS' = \evS \cup \{\evs', \evt'\}$
    \item $\evpr'(\evs) = \evpr(\evs)$ if $\evs \in \evS$
    \item $\evpr'(\evs') = \phi$
    \item $\evpr'(\evt') = \top$
    \item $\evR'_a = \evR_a \cup \{ (\evs', \evt) | \evt \in \evT \text{ and }
      \evt \evR_a \evt \} \cup \{(\evs', \evt')\}$ for all $a \in A$
  \end{itemize}
  This operation {\em joins} a number of action points together, based upon what
  an agent already considers possible.
\end{defn}

This operation is semantically hard to justify/make meaning of.
However, I think it does ascribe to the ``design philosophy" James suggested, in
the sense that it is adding relations to new states only.\\
\\
I believe this operation will be fine for ensuring we can build all
$\axK\axD\axFo\axFi$ action models, up to $n$-bisimilarity, but I am not
convinced (or unsure) that it will build only $\axK\axD\axFo\axFi$ action models
at this stage.

\end{document}
