\chapter{Conclusion}

We have shown methods to compose action models from smaller ones in two different axiom systems --- $\AXK$
and $\AXKFF$.
We have employed a set of meaningful operations in both systems for action model composition.
In both systems, we have shown that our languages ($\EM(\to,\sqcup)$ in $\AXK$ and
$\EM(\to,\sqcup,I)$ in $\AXKFF$) are complete with respect to all action models, up to
$n$-bisimilarity.\\
\\
In addition, we have shown complimentary results with regards to epistemic goal achievement in both
of our axiomatic systems.
We note that
\begin{itemize}
	\item In $\AXK$ we can construct an action model that will achieve an epistemic goal
	\item In $\AXKFF$, if an action model exists that achieves the epistemic goal, we can construct a
		$\AXKFF$ action model that also achieves this goal
\end{itemize}

Our results are novel with regards to the ideas behind action model composition.
In particular, we unify the strong semantics behind action models with a syntactical approach reminiscent of van
Ditmarsch's \cite{ditmarsch2002dga} epistemic relational actions.
This unification provides some of the strengths of both approaches, whilst maintaining the
meaningful semantics of action model updates.\\
\\
Beyond the semantics and syntax of action models, this paper contributes towards action model
composition frameworks.
To the author's knowledge, this is a novel approach to action models and contributes toward ensuring
that updates can be constructed in a logical fashion and in a manner that preserves the consistency
of the knowledge base.
This consistency is desirable in epistemic games, model checking and security, where reasoning about
the change in knowledge of agents is a useful ability.\\
\\
We have yet to further explore how succinct our updates are, in terms of their size or the changes
they make.
Furthermore, even if we can construct different kinds of updates, what updates are the most useful?
Which action models or updates might we be most interested in constructing, and if so are there
better composition methods that can construct them?\\
\\
This approach is thus rudimentary, but appear to be one of the first steps in an unexplored area.
These constructions yield all action models, up to $n$-bisimilarity, and provide the first necessary
tools towards employing action models and the updates they represent in dynamic automated
multi-agent systems.
% Every research paper should answer the following questions:
% 
% \begin{itemize}
% \item What did you do?
% \item Why did you do it?
% \item What happened?
% \item What do the results mean?
% \item What is your work good for?
% \end{itemize}
% 
% Make sure that your conclusion leaves the reader with the answers
% to these questions clearly in mind.
% 
