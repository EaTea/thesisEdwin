\chapter{Conclusion}

We have shown methods to construct epistemic event models (that is, epistemic updates) in two different axiom systems --- $\AXK$
and $\AXKFF$.
We have employed a set of meaningful operations in both systems in order to construct event models.
In doing so, we have also shown that if there is an appropriate epistemic goal, then
\begin{itemize}
	\item in $\AXK$ we can construct an event model that will achieve that epistemic goal
	\item in $\AXKFF$, if an update model exists that achieves the epistemic goal, we can construct an
$\AXKFF$ event model that will have also achieve this goal
\end{itemize}

It is important to note that event models, their synthesis, and their practical applications in
developing protocols have not been well examined.
Our work has specified operations that are meaningful in their axiomatic systems, and we have proven
that for some event model $\evM_\evT$ we can use our operations to form event models that will
achieve the same post-conditions as $\evM_\evT$ after execution.
This is a first step in employing event models as updates in knowledge base systems, especially
games, model checking and financial systems.
This is due to our operations provide a high-level, theoretical way to construct updates which we know are correct.
Furthermore, we know that the updates we can construct are complete --- that is, they can achieve
all the same epistemic goals as other updates.\\
\\
We have yet to further explore how succinct our updates are.
Can we make the smallest update?
Can we make update that changes the knowledge base in the ``smallest way"?
Furthermore, even if we can construct so many different kinds of updates, what updates are the most
useful?
Which event models or updates might we be most interested in constructing, and if so are there
better methods of constructing them?\\
\\
The work in this paper outlines an approach to event model construction that is rudimentary, but is,
to the author's knowledge, one of the first steps in an unexplored area.
Event models, as updates for which properties can be both proven and reasoned about, are powerful
constructs.
By having event model construction tools, we have the first steps towards employing event models and
the updates they represent in dynamic automated multi-agent systems.

% Every research paper should answer the following questions:
% 
% \begin{itemize}
% \item What did you do?
% \item Why did you do it?
% \item What happened?
% \item What do the results mean?
% \item What is your work good for?
% \end{itemize}
% 
% Make sure that your conclusion leaves the reader with the answers
% to these questions clearly in mind.
% 
