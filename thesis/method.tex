\section{Methodology}

Lots of background material~\cite{Du}, explanations of the
theory~\cite{Fourier}, how the algorithm works, and what the
appropriate theorems are etc. goes here.

This is a section in which you are likely to use various mathematical
symbols, as you describe the theory behind your work. You can put
maths into a sentence very simply, for example $x = y^2 - 2$, or
you can create an equation as follows:

\begin{equation}
x = y^2 - 2
\end{equation}

Don't forget that mathematical symbols and equations form part of the sentence
structure, so maintain normal English syntax and grammar throughout
your constructions.

Here might be a good place to put in a diagram or two, although \LaTeX\/
might not necessarily put them right here!

\begin{figure}[htbp]
\par
\centerline{\psfig{figure=vdu.ps}}
\par
\caption{Figure 1 is here and it illustrates an important point in the paper.
Notice that captions are proper constructs in English too---they start with
capital letters and end with full stops.}
\end{figure}

