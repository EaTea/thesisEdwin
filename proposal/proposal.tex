\documentclass[12pt, a4paper]{article}
\usepackage {amsfonts}
\setlength{\oddsidemargin}{0.5cm}
\setlength{\evensidemargin}{0.5cm}
\setlength{\topmargin}{-1.6cm}
\setlength{\leftmargin}{0.5cm}
\setlength{\rightmargin}{0.5cm}
\setlength{\textheight}{24.00cm} 
\setlength{\textwidth}{15.00cm}
\parindent 0pt
\parskip 5pt
\pagestyle{plain}

\title{Research Proposal --- Bachelor of Software Engineering}
\date{14th March 2013}

\newcommand{\namelistlabel}[1]{\mbox{#1}\hfil}
\newenvironment{namelist}[1]{%1
\begin{list}{}
    {
        \let\makelabel\namelistlabel
        \settowidth{\labelwidth}{#1}
        \setlength{\leftmargin}{1.1\labelwidth}
    }
  }{%1
\end{list}}

\begin{document}
\maketitle

\begin{namelist}{}
\item[{\bf Title:}]
	A formal language for building informative updates and a process (via
  action models) for executing them
  \item[{\bf Author:}]
	Edwin Tay, 20529864
\item[{\bf Supervisor:}]
	Associate Professor Tim French
\end{namelist}

\section*{Aim}
%Now state explicitly the hypothesis you aim to
%test. Make references to the items listed in the Reference section
%that back up your arguments for why this is a reasonable
%hypothesis to test, for example the work of Knuth~\cite{knuth}.
%Explain what you expect will be accomplished by undertaking this
%particular project.  Moreover, is it likely to have any other
%applications?

Epistemology (the study of knowledge) presents a useful way to specify the possession and transfer of knowledge.
Its representation has become integral to computational applications such as natural language processing and artificial intelligence.
Epistemic models have been developed to describe multi-agent systems that have found usage in games and economics.
Share trading and auctions are just some of the areas where these models are being developed and further improved.\\
\\
Action models are one way to specify how a multi-agent system is to be updated.
They encapsulate the uncertainty in taking an action and when executed on a multi-agent system, update it to a new state of knowledge.
Recent investigations have synthesised an action model to update a pre- to a post-execution state of knowledge.
These executions appear to involve an expensive number of operations to naively implement for a multi-agent system.\\
\\
We propose an investigation into a formal language that can specify an informative update for a multi-agent system.
Our aim is to be able to transform an action model into a series of local informative updates for a single agent of the system.
To test the usefulness of our specification we will explore a message passing implementation of this language to update multiple agents.
This will allow us to evaluate and discuss the advantages, flaws and further development we can propose for this specification.
%
%\begin{itemize}
%  \item Discuss action model updates
%  \item Discuss taking an action model that updates many local agents
%  \item Outline how we will derive a series of local updates for agents
%  \item Outline how we this will be implemented
%  \item Outline how we'd like to implement the model
%  \item We want to investigate whether this is a feasibly computed model
%  \item We want to investigate whether an intermediate specification of updates
%    makes action models useful
%\end{itemize}

\section*{Background}
%In this section you should give some background to your
%research area. What is the problem you are tackling, and why is it
%worthwhile solving? Who has already done some work in this area,
%and what have they achieved?

Epistemology, in the philosophical sense, concerns the study of knowledge.
Answering questions and issues about the nature, potential and semantics of knowledge have been explored since the times of Ancient Greece.
To answer these questions in an objective fashion, we turn to epistemic logic, the logic of knowledge.
Epistemic logic provides axiomatisations of knowledge that capture its properties.
We can formally reason about an agent's knowledge, its knowledge of its knowledge and (in a system with multiple agents) other agents' knowledge.\\
\\
Agents can be be uncertain about the value for a proposition, resulting in multiple worlds that are consistent with the state of the world.
This ``multiple worlds" agent model is a description of the Kripke model that can be described succintly by a Kripke model.\footnote{Kripke models can describe more logics than just epistemic logic, depending upon the properties of the relation $R$.}
Kripke models are theoretical constructs that describe for a set of agents $A = \{a_1, a_2, \ldots\}$ and propositions $P = \{p_1, p_2, \ldots \}$
\begin{itemize}
  \item a finite set of states, $S$
  \item a function $R: A \to S \times S$ that yields a relation between states (the accessibility relation for an agent $a$)
  \item a valuation function $V: P \to \mathbb{P}(S)$ that yields for a proposition $p$ the subset of $S$ where $p$ is true
\end{itemize}
Kripke models can be written as a tuple $(S, R, V)$.\\
\\
Kripke models can be updated via informative updates, which change the knowledge state of an agent or agents.
An update consists of some action that produces, from model $M = (S, R, V)$ an updated model $M' = (S', R', V')$.
We can encode our action into an action model.
Action models resemble Kripke models in their structure, and consist of a tuple $A = (\bar(S), \bar(R), \bar(pre))$ such that
\begin{itemize}
  \item $\bar{S}$ is the set of action model states
  \item $\bar{R}$ is the accessibility relations that for that action; semantically, they represent the uncertainty in taking an action
  \item $\bar{pre}$ is a function that maps a state $s \in \bar{S}$ to a propositional formula that represents the preconditions for that action state to be fulfilled
\end{itemize}
The execution of an action model $A$ to update a Kripke model $M$ to produce $M'$ is a well-understood process.
An update involves taking the Cartesian Product $\bar{S} \times S$ to form the states of $M'$.
The number of states then dramatically decreases by reduction of the valid states.
This presents a problem in utilising action models to represent an informative update in a computer, due to the expensive nature of an operation like the Cartesian product.
%
%\begin {itemize}
%  \item What is epistemic logic?
%  \item What is a Kripke model?
%  \item What is an informative update?
%  \item What is an action model?
%  \item What are the problems with action models?
%\end {itemize}
 
\section*{Method}
%In this section you should outline how you intend to go
%about accomplishing the aims you have set in the previous
%section. Try to break your grand aims down into small,
%achievable tasks. Try to estimate how long you will
%spend on each task, and draw up a timetable for each
%sub-task.

To begin with, we propose a review of action models and their updates.
We would like to classify action models in terms of the kind of update they describe.
From this review of action models, we would aim to achieve
\begin{itemize}
	\item a basic classification scheme of informative updates from action models
	\item examples of how action models update systems of multiple agents
\end{itemize}

Such a classification would aid us in constructing a formal language of agent updates that can build an informative update.
This would involve the use of a $\pi$-calculus to describe agent updates.
In undertaking this, we will have a formal language to describe an informative update in terms of local updates to a single agent.
This allows us to take an action model describing an informative update and use it to theoretically build a series of local updates.\\
\\
The formal specification will then be verified.
We will attempt to prove that both specifications will update the same model to the same final state.
In doing so, we can also compare both of our update specifications (pure actions models versus formal language) against each other using our initially reviewed models.\\
\\
Our final stage will be the implementation and testing of our specification as a message passing system.
We will attempt to model these updates that we first reviewed and test how useful our system is to specify complex updates.
This allows us to answer the question of how useful action models are as an informative update.
We can also discuss how expensive it is to implement an action model execution.
In particular, we can explore what benefits our formal language as as a series of informative updates to specify a single, multi-agent informative update.

%\begin{enumerate}
%  \item list out different action models, exploring how they update
%    different agents (see Baltag and Moss for a similar exercise)
%  \item specify a model in $\pi$-calculus to update agents
%  \item try and verify model's correctness in replicating action model update
%  \item implement model
%  \item demonstrate its results with the initially explored action models
%  \item test models and find out disadvantages and advantages of models
%\end{enumerate}
%
\section*{Software and Hardware Requirements}
%Outline what your specific requirements will be with regard
%to software and hardware, but note that any special requests
%might need to be approved by your supervisor and the Head of
%Department.
At this time, my project does not appear to require any software or hardware.
Instead, a whiteboard, some whiteboard cleaning liquid and full whiteboard markers would be much appreciated.

\section*{Timeline}

\begin{center}
	\begin{tabular}{ | l | p{5cm} | p{5cm} |}
		\hline
		Date & Task & Description \\
		\hline
		14 March 2013 & Hand in project proposal/summary & The project proposal and summary will be due today.\\
		\hline
		25 April 2013 & Hand in draft Literature Review and Project Proposal & A literature review should be finished; this should allow me time to discuss and edit my project proposal as see fit.\\
		\hline
		2 May 2013 & Construct a draft introduction & The introduction is based mostly off the project proposal; it should succintly describe the project purpose and background.\\
		\hline
		16 May 2013 & Re-edit literature review & Edit the literature review to be in line with the new proposal.\\
		\hline
		23 May 2013 & Resubmit literature review & For further marking and analysis\\
		\hline
		June 2013 & & Construct a specification for informative updates \\
		\hline
		July 2013 & & Write about the specification for informative updates \\
		\hline
		July 2013 & & Prove the specification for informative updates \\
		\hline
		Early August 2013 & & Design and implement message passing informative update \\
		\hline
		Early August 2013 & & Write about message passing and its implementation/design details \\
		\hline
		Late August 2013 & & Evaluate the implementation and whether action models are still useful informative updates\\
		\hline
		Early September 2013 & & Begin writing about specification and the results we have achieved \\
		\hline
		19 September 2013 & & Draft thesis due \\
		\hline
	\end{tabular}
\end{center}

%\begin{enumerate}
%  \item literature review
%  \item example specification
%  \item model verification/proof
%  \item model implementation outline and design
%  \item implementation plan
%  \item implementation development
%  \item implementation testing
%  \item paper writeup
%\end{enumerate}

%Overall, you should aim to produce roughly a two page document
%(and certainly no more than four pages)
%outlining your plan for the year.

\end{document}
