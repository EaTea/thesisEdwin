\documentclass[12pt, a4paper]{article}
\setlength{\oddsidemargin}{0.5cm}
\setlength{\evensidemargin}{0.5cm}
\setlength{\topmargin}{-1.6cm}
\setlength{\leftmargin}{0.5cm}
\setlength{\rightmargin}{0.5cm}
\setlength{\textheight}{24.00cm} 
\setlength{\textwidth}{15.00cm}
\parindent 0pt
\parskip 5pt
\pagestyle{plain}

\title{Research Proposal --- Bachelor of Software Engineering}
\date{14th March 2013}

\newcommand{\namelistlabel}[1]{\mbox{#1}\hfil}
\newenvironment{namelist}[1]{%1
\begin{list}{}
    {
        \let\makelabel\namelistlabel
        \settowidth{\labelwidth}{#1}
        \setlength{\leftmargin}{1.1\labelwidth}
    }
  }{%1
\end{list}}

\begin{document}
\maketitle

\begin{namelist}{}
\item[{\bf Title:}]
	A formal language for building informative updates and a process (via
  action models) for executing them
  \item[{\bf Author:}]
	Edwin Tay, 20529864
\item[{\bf Supervisor:}]
	Associate Professor Tim French
\end{namelist}

\section*{Background}
%In this section you should give some background to your
%research area. What is the problem you are tackling, and why is it
%worthwhile solving? Who has already done some work in this area,
%and what have they achieved?

Epistemology, in the philosophical sense, concerns the study of knowledge.
Answering questions and issues about the nature, potential and semantics of knowledge have been explored since the times of Ancient Greece.
To answer these questions in an objective fashion, we turn to epistemic logic, the logic of knowledge.
Epistemic logic provides axiomatisations of knowledge that capture its properties.
We can formally reason about an agent's knowledge, its knowledge of its knowledge and (in a system with multiple agents) other agents' knowledge.\\
\\
Agents can be be uncertain about the value for a proposition, resulting in multiple worlds that are consistent with the state of the world.
This ``multiple worlds" agent model is a description of the Kripke model that can be described succintly by a Kripke model. \footnote{Kripke models can describe more logics than just epistemic logic, depending upon the properties of the relation $R$.}
Kripke models are theoretical constructs that describe for a set of agents $A = \{a_1, a_2, \ldots\}$ and propositions $P = \{p_1, p_2, \ldots \}$
\begin{itemize}
  \item a finite set of states, $S$
  \item a function $R: A \to S \times S$ that yields a relation between states (the accessibility relation for an agent $a$)
  \item a valuation function $V: P \to \mathbb{P}(S)$ that yields for a proposition $p$ the subset of $S$ where $p$ is true
\end{itemize}
Kripke models can be written as a tuple $(S, R, V)$.\\
\\
Kripke models can be updated via informative updates, which change the knowledge state of an agent or agents.
An update consists of some action that produces, from model $M = (S, R, V)$ an updated model $M' = (S', R', V')$.
We can encode our action into an action model.
Action models resemble Kripke models in their structure, and consist of a tuple $A = (\bar(S), \bar(R), \bar(pre))$ such that
\begin{itemize}
  \item $\bar{S}$ is the set of action model states
  \item $\bar{R}$ is the accessibility relations that for that action; semantically, they represent the uncertainty in taking an action
  \item $\bar{pre}$ is a function that maps a state $s \in \bar{S}$ to a propositional formula that represents the preconditions for that action state to be fulfilled
\end{itemize}
The execution of an action model $A$ to update a Kripke model $M$ to produce $M'$ is a well-understood but expensive process.
It involves taking the Cartesian Product of 

\begin {itemize}
  \item What is epistemic logic?
  \item What is a Kripke model?
  \item What is an informative update?
  \item What is an action model?
  \item What are the problems with action models?
\end {itemize}

\section*{Aim}
%Now state explicitly the hypothesis you aim to
%test. Make references to the items listed in the Reference section
%that back up your arguments for why this is a reasonable
%hypothesis to test, for example the work of Knuth~\cite{knuth}.
%Explain what you expect will be accomplished by undertaking this
%particular project.  Moreover, is it likely to have any other
%applications?

\begin{itemize}
  \item Discuss action model updates
  \item Discuss taking an action model that updates many local agents
  \item Outline how we will derive a series of local updates for agents
  \item Outline how we this will be implemented
  \item Outline how we'd like to implement the model
  \item We want to investigate whether this is a feasibly computed model
  \item We want to investigate whether an intermediate specification of updates
    makes action models useful
\end{itemize}
 
\section*{Method}
%In this section you should outline how you intend to go
%about accomplishing the aims you have set in the previous
%section. Try to break your grand aims down into small,
%achievable tasks. Try to estimate how long you will
%spend on each task, and draw up a timetable for each
%sub-task.

\begin{enumerate}
  \item list out different action models, exploring how they update
    different agents (see Baltag and Moss for a similar exercise)
  \item specify a model in $\pi$-calculus to update agents
  \item try and verify model's correctness in replicating action model update
  \item implement model
  \item demonstrate its results with the initially explored action models
  \item test models and find out disadvantages and advantages of models
\end{enumerate}

\section*{Software and Hardware Requirements}
%Outline what your specific requirements will be with regard
%to software and hardware, but note that any special requests
%might need to be approved by your supervisor and the Head of
%Department.

\begin{itemize}
  \item nothing$\ldots$?
\end{itemize}

\section*{Timeline}
%What even the fuck.

\begin{enumerate}
  \item literature review
  \item example specification
  \item model verification/proof
  \item model implementation outline and design
  \item implementation plan
  \item implementation development
  \item implementation testing
  \item paper writeup
\end{enumerate}

%Overall, you should aim to produce roughly a two page document
%(and certainly no more than four pages)
%outlining your plan for the year.

\end{document}
