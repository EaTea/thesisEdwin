%-----------------------------------------------------------------------------%
%Packages%
\documentclass{beamer}
\usepackage{tikz}
\usepackage{comment}

\usetikzlibrary{arrows,shapes,automata}

% Drawings of frames

\tikzstyle{vertex}=[circle,fill=black!25,minimum size=30pt,inner sep=0pt]
\tikzstyle{selected vertex} = [vertex, fill=red!24]
\tikzstyle{edge} = [draw,thick,->]
\tikzstyle{weight} = [font=\small]

\pgfdeclarelayer{background}
\pgfsetlayers{background,main}

%-----------------------------------------------------------------------------%
%Document%
\begin{document}
\title{A Logical Framework for realising Action Model Execution}
\author{Edwin Tay}
\date{19th April 2013}

\begin{comment}
My honours presentation.
Need to discuss project and give a motivating idea.
In general:

given an epistemic goal we can find an action model we can construct an
action model

given an epistemic action model how can we realise it

do no maths, use pictures

don’t say “this is too complicated” - don’t
mention the “too complicated stuff”!
\end{comment}

\frame{\titlepage}

\begin{frame}
\frametitle{Wanna play a game$\ldots$?}
\begin{itemize}
  \item Consider two agents, Amy ($A$) and Brian ($B$)
  \item There's a coin in front of them, but they don't know whether the coin is
    Heads ($H$) or Tails ($T$)$\ldots$
  \item We can model this as $\ldots$
\end{itemize}
\end{frame}

\begin{frame}
\frametitle{As a Kripke Structure}
\begin{figure}
\begin{tikzpicture}[->,>=stealth',shorten >=1pt,auto,node distance=3cm,
      thick]

    \node[vertex] (1) {$H$};
    \node[vertex] (2) [right of=1] {$T$};
    \path[edge]
          (1) edge node {$A,B$} (2)
              edge [loop left] node {$A,B$} (1)
          (2) edge node {} (1)
              edge [loop right] node {$A,B$} (2)
\end{tikzpicture}
\end{figure}
\end{frame}

\begin{frame}
\frametitle{Suspicion!}
\begin{itemize}
  \item Amy thinks Brian knows more than he's letting on
  \item It's possible Brian knows whether the coin is Heads or Tails
  \item Or maybe he doesn't know any more than Amy knows
  \item But Brian hasn't actually cheated
\end{itemize}
\end{frame}

\begin{frame}
\frametitle{Suspicion: A Kripke Structure}
\begin{figure}
\begin{tikzpicture}[->,>=stealth',shorten >=1pt,auto,node distance=3cm,thick]

    \node[vertex] (1) {$H$};
    \node[vertex] (2) [right of=1] {$T$};
    \node[vertex] (3) [below of=1] {$H'$};
    \node[vertex] (4) [below of=2] {$T'$};
    \path[edge]
          (1) edge node {$A,B$} (2)
              edge [loop left] node {$A,B$} (1)
              edge node {$A$} (3)
              edge node {$A$} (4)
          (2) edge node {} (1)
              edge [loop right] node {$A,B$} (2)
              edge node {$A$} (3)
              edge node {$A$} (4)
          (3) edge node {$A$} (4)
              edge [loop left] node {$A,B$} (3)
              edge node {} (1)
              edge node {} (2)
          (4) edge node {} (3)
              edge [loop right] node {$A,B$} (4)
              edge node {} (1)
              edge node {} (2)
\end{tikzpicture}
\end{figure}
\end{frame}

\begin{frame}
\frametitle{Suspicion: A Kripke Structure}
\begin{figure}
\begin{tikzpicture}[->,>=stealth',shorten >=1pt,auto,node distance=3cm,thick]

    \node[vertex] (1) {$H$};
    \node[vertex] (2) [right of=1] {$T$};
    \node[vertex] (3) [below of=1] {$H'$};
    \node[vertex] (4) [below of=2] {$T'$};
    \path[edge]
          (1) edge node {$A,B$} (2)
              edge [loop left] node {$A,B$} (1)
              edge[red!88] node {$A$} (3)
              edge[red!88] node {$A$} (4)
          (2) edge node {} (1)
              edge [loop right] node {$A,B$} (2)
              edge[red!88] node {$A$} (3)
              edge[red!88] node {$A$} (4)
          (3) edge[red!88] node {$A$} (4)
              edge[loop left, red!88] node {$A,B$} (3)
              edge[red!88] node {} (1)
              edge[red!88] node {} (2)
          (4) edge[red!88] node {} (3)
              edge[loop right, red!88] node {$A,B$} (4)
              edge[red!88] node {} (1)
              edge[red!88] node {} (2)
\end{tikzpicture}
\end{figure}
\end{frame}
\begin{frame}
\frametitle{The state of information}
\begin{itemize}
  \item Describing the state of information is quite doable at any situation
  \item Problem: how do we move from one situation to another?
  \item How do we specify an action that changes situation from one to another?
\end{itemize}
\end{frame}

\begin{frame}
\frametitle{Action Models}
\begin{itemize}
  \item Mathematical specification of an action
  \item Describes how to change the state of information from pre- to post-change
  \item Essentially a Kripke Model of an action
\end{itemize}
\end{frame}

\begin{frame}
  \frametitle{Problems with Action Models}
\begin{itemize}
  \item Action Models describe change in information, but how is that information distributed?
  \item What can we use to compose or construct action models?
  \item What are the ``atomics"? Can we generate any action model using some predefined atomic structures?
  \item Will Action Models actually be useful or practical to execute?
\end{itemize}
\end{frame}

\begin{frame}
\frametitle{Project Aims}
\begin{itemize}
  \item Explore whether a generating set exists for action models/classes of action models
  \item Explore what operations might be needed to perform this generation
  \item Use pointed action model execution to specify a per-agent model of an entire action model update
  \item Measure the complexity of such an execution
\end{itemize}
\end{frame}

\end{document}
