%-----------------------------------------------------------------------------%
%Packages%
\documentclass[10pt, a4paper, twoside]{article}
\usepackage{amsmath, amsfonts, listings, amssymb, mathtools, amsthm} %Mathematical Expressions package
\usepackage{mathtools}
\usepackage[usenames, dvipsnames]{color} %Color naming packages
\usepackage[margin=1.5cm]{geometry}
\usepackage{float}
\usepackage{verbatim} %for code
\usepackage[pdftex]{graphics}
\usepackage{ulem}
\usepackage{hyperref}
\usepackage{cleveref}
\usepackage{thmtools}
\usepackage{tikz}
\usepackage{comment}

\usetikzlibrary{arrows,shapes}

%Graphis Extensions
\DeclareGraphicsExtensions{.png, .jpg}
\parindent 0pt

% Predefined things such as commands, etc.

\newcommand{\aRel}[1] {
  \sim_{#1} 
}

\newcommand{\kripkeFrame}[2] {
  (#1, \aRel{#2})
}

\newcommand{\kripkeModel}[3] {
  (#1, \aRel{#2}, #3)
}

\newcommand{\frKripModel}[2] { % defined via Kripke Frame + valuation
  (#1, #2)
}

\newcommand{\actModel}[3]{
  (#1, \aRel{#2}, #3)
}

\newcommand{\frActModel}[2] { % defined via Kripke Frame + Pre
  (#1, #2)
}

\newcommand{\note}[1]{\textsc{\textbf{#1}}}
\newcommand{\Universal}{$\mathcal{U}$}
\newcommand{\modalLog}{$\mathcal{L}$}
\newcommand{\modLogInf}{$\mathcal{L}_\inf$}
\newcommand{\epActLog}{\modalLog$([\alpha])$}
\newcommand{\epActLogCommonKnowledge}{\modalLog$([\alpha],\box^{*})$}

\newtheorem{defn}{Definition}
\newtheorem{thm}{Theorem}
\newtheorem{lemma}{Lemma}
\newtheorem*{remrk}{Remark}

% Drawings of frames

\tikzstyle{vertex}=[circle,fill=black!25,minimum size=20pt,inner sep=0pt]
\tikzstyle{selected vertex} = [vertex, fill=red!24]
\tikzstyle{edge} = [draw,thick,-]
\tikzstyle{weight} = [font=\small]

%-----------------------------------------------------------------------------%
%Document%
\begin{document}

\begin{comment}
The literature review should be about 3000-5000 words long.
I would probably aim for around 4000.
You'll need about 5 to 10 papers. Aim for 10?
\end{comment}

\tableofcontents

\vfill
\pagebreak


\section{Introduction}\label{intro}
\begin{itemize}
  \item A motivation for this study; to define a framework for epistemic actions
  using minimal ``atoms"
  \item Provide a basis for exploring different updates and dynamic epistemic
  logic
  \item Proposed direction would be to model epistemic updates in the context of
  message passing systems
  \item Need to provide a clear motivation for study in this area
  \item Have to show {\em why} we'd want to specify and realise epistemic
  updates
\end{itemize}

\section{Epistemic Logic: modelling information}\label{epistemics}
\begin{itemize}
\item Give some background into modal logic, epistemic logic and dynamic
epistemic logic.
\item Essentially, what is modal logic? What is epistemic logic? What is dynamic
epistemic logic?
\item might be a good idea to define some concepts we'd talk about such as box,
  diamond, kripke frames
\item Formalisations of concepts in introduction: perhaps not necessary?
\item Reference DEL, Modal Logic here
\end{itemize}

\section{Extending PDL}\label{pdlExtend}
\subsection{Definition of PDL} \label{pdlDefine}
\begin{itemize}
  \item Use the first introduction of PDL (old paper from 1979?)
  \item Motivate why PDL is useful as a specification language
  \item What does PDL allow us to reason about?
  \item What does PDL NOT allow us to reason about?
  \item Why is PDL a good guiding framework?
\end{itemize}
\subsection{Message passing with PDL} \label{pdlMessages}
\begin{itemize}
\item discuss Bollig et. al's work (Propositional Dynamic Logic for
    Message-Passing Systems)
\item what is the significance of the work? essentially given an MSC to specify
a sequence of messages we can translate this to a formula in PDL
\item a finite state machine that executes this MSC is of exponential size
relative to the number of propositions
\item comment on how this suggests that updates themselves will be expensive
\end{itemize}

\section{Public Announcements}\label{announcements}
\begin{itemize}
\item discuss public announcements as a singular entity
\item mentioned in Baltag, Moss, Solecki and in DEL
\item what kind of an update do they entail
\item what can't we do with them?
\item what does this afford us as an advantage compared to PDL?
\item what have we lost in moving to this representation?
\end{itemize}

\section{Action Models}\label{actModels}
\begin{itemize}
  \item reiterate the weaknesses of what we've seen so far and wonder if we can
  cover them
  \item introduce the concept of an action model
\end{itemize}
\subsection{Action Models Definition}\label{actModelDefn}
\begin{itemize}
  \item first covered in BMS1998
  \item give the definition of a model as a Kripke frame with a valuation
  \item motivate the model itself, the reasoning you can do with it
  \item BMS claim you can model message passing with their constructs, but they
  give no ideas about it
  \item furthermore it's all about what the omniscient 3rd party sees, not what
  the agents themselves can see
\end{itemize}
\subsection{Extending PDL}\label{actModelExtends}
\begin{itemize}
  \item BM2004 suggests an epistemic program, similar to PDL
  \item gives us a powerful syntax similar to PDL
  \item what are the consequences?
  \item what can we express, and what updates can't we express?
  \item comparison of using PDL for message passing and the consequences of this
  \item notice that we're not sure what the smallest set of things to generate
  all action models are
  \item we don't know how we can realise action models
  \item if PDL is anything to go by we could be using exponential states to
  communicate things, is not good
\end{itemize}

\section{Conclusion}\label{conclusion}
\begin{itemize}
  \item Resummarise what this literature review has discussed
  \item Make any comparisons and comments upon whether we can realise an
  epistemic update
  \item remotivate my topic; no-one has yet thought about how the agents can
  realise a specified update and what the complexity of this is
\end{itemize}

\end{document}
